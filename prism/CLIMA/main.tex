\documentclass{llncs}

\usepackage{llncsdoc}
\usepackage{graphicx}
\usepackage{float}
\usepackage{algorithmicx}
\usepackage{algorithm}
\usepackage{algpseudocode}
\usepackage{amsmath}
\usepackage{soul}
\usepackage{color}


\begin{document}

\title{Verifying Team Formation Protocols in PRISM\thanks{This work is supported by
the ERC Advanced Grant VERIWARE.}}
\author{}
\institute{
  Computing Laboratory, University of Oxford\\
  Wolfson Building, Parks Road, Oxford, OX1 3QD, UK
}
\email{}

\maketitle

\begin{abstract}

\end{abstract}

\section{Introduction}


\section{Preliminaries}

\subsection{Discrete-Time Markov Chain}
\subsection{Markov Decision Process}
\subsection{Stochastic Two-Player Game}

\section{Protocol Modeling and Verification}

\begin{definition}
An agent \textbf{organisation} is a tuple $O=\langle A, N, R, R_A  \rangle$ where  
\noindent
\begin{itemize}
\item $A= \{ a_1,a_2,...,a_n \} $ is a set of agents,
\item $N=\{\{a_i,a_j \}\ :$ ``$a_i$ and $a_j$ are neighbours'' $\}$ is a neighbourhood relation,
\item $R=\{r_1,r_2,\dots,r_k\}$ is a set of resource types,
\item $R_A=\{R_{a_1}, R_{a_2}, \dots, R_{a_n}\}$ is a set of agent resources where $r_j \in R_{a_i} $ $\iff$ ``agent $a_i$ has a resource $r_j$''.  
\end{itemize}
\end{definition}

\begin{definition}
A \textbf{task} $T_i=\{r_i\ :\ $``$r_i$ is required by the task $i$''$\}$ is a set of resources that are required to complete it. By $T=\{T_1, T_2,\dots, T_t\}$ we denote a collection of tasks.
\end{definition}

\begin{definition}
A \textbf{team} of agents is denoted by $M_i=\{a_i\ :\ $``$a_i$ is a member of team $i$''$\}$, and the set of all teams is $M=\{M_1, M_2,\dots , M_m\}$. By $\bar{M} = \bigcup_{M_i \in M} M_i$ we denote the set of all agents that are committed to some team. $R_{M_i}=\bigcup_{a \in M_i}R_{a}$ is the set of resources that the team $M_i$ has. The team $M_i$ is said to be able to complete the task $T_j$ iff $T_j \subseteq R_{M_i}$.
\end{definition}

\begin{definition}
Team \textbf{initiation probability} for agent $a$ is the ratio between neighbours that are not committed to any team and total number of neighbours:

\begin{equation}
IP_a = \frac{|\{ a' \in A\ :\ \{a, a'\} \in N \wedge a' \notin \bar{M}  \}|}{|\{ a' \in A\ :\ \{a, a'\} \in N \}|}.
\label{eq:init_prob}
\end{equation} 
 
\end{definition}

\begin{definition}
We define two types of \textbf{reward} for agent $a$. Type $W_1$ rewards agent with 1 point if it is in the team which was able to complete its task after team formation is over or 0 otherwise.Type $W_2$ rewards 1 point to the team which was able to complete its task, and 0 otherwise and the reward is shared equally between team members,

\begin{equation}
W_1(a) = \left\{
  \begin{array}{l l}
    1 & \quad \text{if $\exists M_i . a \in M_i \wedge T_i \subseteq R_{M_i}$,}\\
    0 & \quad \text{otherwise,}\\
  \end{array} \right. 
\label{eq:w1agent}
\end{equation}
\begin{equation}
W_2(a) = \left\{
  \begin{array}{l l}
    \frac{1}{|M_i|} & \quad \text{if $\exists M_i . a \in M_i \wedge T_i \subseteq R_{M_i}$,}\\
    0 & \quad \text{otherwise.}\\
  \end{array} \right. 
\label{eq:w2agent}
\end{equation}

The rewards of an agent organisation $O$ are defined accordingly as the total reward achieved by its members,
\begin{equation}
 W_1(O) = \sum_{a \in A} W_1(a),
\label{eq:w1organisation}
\end{equation}
\begin{equation}
 W_2(O) = \sum_{a \in A} W_2(a).
\label{eq:w2organisation}
\end{equation}

\end{definition}



\begin{algorithm}[H]
\caption{Offline and online versions of the algorithm}
\label{alg:main_process}
\begin{scriptsize}
\begin{algorithmic}
\Procedure{Offline}{$t$} \Comment{$t$ - number of tasks}
  \State $M = \{M_i = \emptyset\ :\ 1\leq i \leq t\}$ \Comment{initialise empty teams}
  \State $T = \{T_i\neq \emptyset\ :\ T_i \subseteq_{random} R,\ 1\leq i \leq t\}$ \Comment{initialise tasks at random}
  \ForAll {$a \in A$ in random order}
    \State \Call{JoinTeam}{$a$, $T$, $M$}
  \EndFor
  \State perform tasks and compute rewards
\EndProcedure
\Statex
\Procedure{Online}{$t$} \Comment{$t$ - number of tasks}
  \State $M = \{M_i = \emptyset\ :\ 1\leq i \leq t\}$ \Comment{initialise empty teams} 
  \ForAll {$a \in A$ in random order}
    \State \Call{JoinTeam}{$a$, $T$, $M$}
  \EndFor
  \State $T = \{T_i\neq \emptyset\ :\ T_i \subseteq_{random} R,\ 1\leq i \leq t\}$ \Comment{initialise tasks at random}
  \State perform tasks and compute rewards
\EndProcedure

\end{algorithmic}
\end{scriptsize}
\end{algorithm}

\begin{algorithm}[H]
\caption{Team joining algorithm (probabilistic and deterministic)}
\label{alg:join_team_org}
\begin{scriptsize}
\begin{algorithmic}
\Procedure{JoinTeam}{$a$, $T$, $M$}
  \ForAll {$T_i \in T$ in random order} 
    \If {$a \notin \bar{M}$} \Comment{agent is not committed}
      \If {$|M_i|=0$} \Comment{team for task i is empty}
	\If {$R_a \cap T_i \neq \emptyset$} \Comment{agent has skill}
	  \State with probability $IP_a$: $M_i \leftarrow M_i \cup \{a\}$ \Comment{initiate a team}
	\EndIf
      \ElsIf {$\exists \{a,a'\} \in N . a' \in M_i$} \Comment{there is neighbour in team for task i}
	\If {$R_a \cap T_i \setminus R_{M_i} \neq \emptyset$} \Comment{agent has a missing resource}
	  \State $M_i \leftarrow M_i \cup \{a\}$ \Comment{join team}
	\EndIf
      \EndIf
    \EndIf
  \EndFor
\EndProcedure
\end{algorithmic}
\end{scriptsize}
\end{algorithm}

\begin{algorithm}[H]
\caption{Team joining algorithm (non-deterministic)}
\label{alg:join_team_nondet}
\begin{scriptsize}
\begin{algorithmic}
\Procedure{JoinTeam}{$a$, $T$, $M$}
  \ForAll {$T_i \in T$ in \hl{arbitrary} order} 
    \If {$a \notin \bar{M}$} \Comment{agent is not committed}
      \If {$|M_i|=0$} \Comment{team for task i is empty}
	\If {$R_a \cap T_i \neq \emptyset$} \Comment{agent has skill}
	    \State \hl{$M_i \leftarrow M_i \cup \{a\}$  \textbf{or} $M_i \leftarrow M_i$} \Comment{initiate a team or do nothing}
	\EndIf
      \ElsIf {$\exists \{a,a'\} \in N . a' \in M_i$} \Comment{there is neighbour in team for task i}
	\If {$R_a \cap T_i \setminus R_{M_i} \neq \emptyset$} \Comment{agent has a missing resource}
	  \State \hl{$M_i \leftarrow M_i \cup \{a\}$ \textbf{or} $M_i \leftarrow M_i$} \Comment{join a team or do nothing}
	\EndIf
      \EndIf
    \EndIf
  \EndFor
\EndProcedure
\end{algorithmic}
\end{scriptsize}
\end{algorithm}



\subsection{DTMC}
\subsubsection{Protocol}
\subsubsection{Model and Specification}
\subsubsection{Results}

\subsection{MDP}
\subsubsection{Protocol}
\subsubsection{Model and Specification}
\subsubsection{Results}

\subsection{STPG}
\subsubsection{Protocol}
\subsubsection{Model and Specification}
\subsubsection{Results}

\section{Conclusions and Future Work}

We would like to outline several directions for future work:
\begin{itemize}

 \item Conducting experiments with agents having multiple resources, allowing agents to change teams until stable point is reached and check whether it can reached, what is the expected number of steps to reach the stable configuration, etc.

\item Verifying dual properties, i.e. rather than asking ``what is the expected reward for a strategy'', we would like to be able to model-check the properties of the form ``whats is the probability of reaching reward value R with a strategy?''.  

 \item In this paper we only considered sequential execution of agent algorithms, in future work we would like to explore parallel execution (i.e. interleaving steps of JoinTeam algorithm for multiple agents), and see what effect this has for agent strategies in both online and offline versions of the algorithm. Also this raises many interesting scheduling problems, as worst and best case scheduling (of action interleaving) scenarios are unrealistic, the challenging question is to both develop realistic schedulers and perform model-checking efficiently. 

 \item Another interesting question is construction of optimal agent organisations given the set of tasks and their distributions.This problem is an instance of mechanism design problem in game-theoretic settings, where the designer can have control over neighbourhood structure or agent resources or both.

\end{itemize}


\end{document} 