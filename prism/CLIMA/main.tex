\documentclass{llncs}

\usepackage{llncsdoc}
\usepackage{graphicx}
\usepackage{float}
\usepackage{algorithmicx}
\usepackage{algorithm}
\usepackage{algpseudocode}
\usepackage{amsmath}
\usepackage{soul}
\usepackage{color}


\begin{document}

\title{Verifying Team Formation Protocols in PRISM\thanks{This work is supported by
the ERC Advanced Grant VERIWARE.}}
\author{}
\institute{
  Computing Laboratory, University of Oxford\\
  Wolfson Building, Parks Road, Oxford, OX1 3QD, UK
}
\email{}

\maketitle

\begin{abstract}

\end{abstract}

\section{Introduction}


\section{Preliminaries}

%% G. Norman and V. Shmatikov.

%Analysis of Probabilistic Contract Signing.
%
%Journal of Computer Security, 14(6), pages 561-589, IOS Press.
%
%November 2006.

\subsection{Probabilistic Model Checking}
Probabilistic model checking is an automated technique for
the formal verification of systems that exhibit stochastic behavior. It can, for example
calculate the likelihood of the occurrence of certain events
during the execution of a system, and can be useful to establish performance
measures. It is based on the construction and analysis of a
mathematical model of the system, usually from a specification
in some high-level description language. This model
generally comprises a set of states, representing all the possible
configurations of the system, the transitions that can
occur between these states, and information about when and
with what probability each transition will occur, and a probabilistic model checker applies algorithmic techniques to
analyze the state space and calculate performance measures.

\subsection{PRISM model checker}

We use PRISM. The current implementation of PRISM supports the analysis of finite state
probabilistic models of the following three types: discrete-time Markov chains,
continuous-time Markov chains and Markov decision processes. These models are
described in a high-level language, a variant of reactive modules [2] based on guarded
commands. The basic components of the language are modules and variables. A
system is constructed as a number of modules which can interact with each other.
A module contains a number of variables which express the state of the module,
and its behavior is given by a set of guarded commands of the form:

[] $<$guard$>$ \! $<$command$>$;

The guard is a predicate over the variables of the system and the command describes
a transition which the module can make if the guard is true (using primed variables
to denote the next values of variables). If a transition is probabilistic, then the
command is specified as:

<prob> : <command> + ...+ <prob> : <command>

PRISM accepts specifications in either PCTL, or CSL logic depending on the model
type. The model checker then analyses the model and checks if the property holds in each
state. 
The reader is invited to consult the PRISM
web site \cite{} for detailed information and corresponding
publications about all of these.

%
%
%In this paper, the modelling formalism we use is discretetime
%Markov chains (DTMCs), where time is modelled as
%discrete steps and the probability of making each transition
%is given by a discrete probability distribution. Other model
%types commonly used are continuous-time Markov chains
%(CTMCs), Markov decision processes (MDPs), and probabilistic
%timed automata (PTAs); see [25] for more detailed information
%about these.We use the probabilistic model checking
%tool PRISM [12, 23]. This allows construction of models
%via specification in a high-level description language, based
%on the parallel composition of several modules described in
%a guarded command notation. We will illustrate the workings
%of this language in more detail later in the paper.
%Models constructed in PRISM are analysed by formally
%specifying formulas in temporal logic. This allows a precise,
%unambiguous description of a wide range of properties, such
%as ��the probability of shutdown occurring within 24 hours��
%or ��the long-run probability that the system is stable��. In
%addition, by assigning real-valued costs (or, conversely, rewards)
%to states and transitions of the model, we can also reason
%about, for example, ��expected time�� or ��expected power
%consumption��.
%PRISM automatically ascertains values for such properties
%by performing probabilistic model checking, which includes
%both graph-based analysis and numerical computation.
%For the case of DTMCs, the latter usually constitutes
%solving a linear equation system of size equal to the number
%of states in the model. For this, PRISM uses iterative numerical
%solution techniques such as the Gauss-Seidel method,
%which are well suited to large problems of this type.
%Furthermore, a significant amount of work has gone into
%the development of efficient, symbolic implementation techniques
%for numerical computation in PRISM. These use data
%structures such as binary decision diagrams (BDDs) [5] and
%their extensions, e.g. multi-terminal BDDs (MTBDDs) [6,
%1], to allow compact storage and manipulation of extremely
%large models. We rely heavily on this efficiency for the case
%study presented in this paper.
%The PRISM tool has already been successfully used to
%perform analysis of, and identify interesting behaviour, in a
%wide range of case studies. This includes the study of ��quality
%of service�� properties for components of real-time probabilistic
%communication protocols, e.g. IEEE 1394 FireWire
%[7, 16], IEEE 802.3 CSMA/CD [17, 8], Zeroconf [13] and
%IEEE 802.11 wireless LANs [15, 24]. It has also been used to
%verify randomised distributed algorithms for leader election,
%consensus [14], Byzantine agreement [10], self-stabilisation
%and mutual exclusion, and probabilistic security protocols
%for anonymity [27], fair exchange and contract signing [20].
%Finally, PRISM has been applied to analysing the performance
%and reliability of many different types of applications:
%dynamic power management schemes [19], NAND
%multiplexing for nanotechnology [18], computer networks,
%queueing systems, manufacturing processes and embedded
%systems [11]. 



\paragraph{Discrete-Time Markov Chain.}

One of basic models used in probabilistic model checking is \emph{discrete time Markov chains} (DTMCs). A DTMC is defined by a 
set of states $S$ and a probability transition matrix
$\mathbf{P} : S\times S \rightarrow [0, 1]$, where
$\sum_{s'\in S} \mathbf{P}(s, s') = 1$ for all $s \in  S$. This gives the probability $\mathbf{P}(s, s')$ that a transition
will take place from state $s$ to state $s'$.


\paragraph{Markov Decision Process}

%M. Duflot, L. Fribourg, T. H��rault, R. Lassaigne, F. Magniette, S. Messika, S. Peyronnet and C. Picaronny.
%
%Probabilistic model checking of the CSMA/CD protocol using PRISM and APMC.


Markov Decision Processes (MDP) allow accurate modelling of systems which exhibit
both probabilistic and nondeterministic behavior. Formally:

A Markov decision process (MDP) is a tuple $(S, s_0, Act, Steps)$ where $S$ is a finite set of states, 
$s_0$ is the initial state, $Act$ is a set of actions
and $Steps  S �� Act �� Dist(S) is a probabilistic transition relation.
%An MDP transition s
%a,��
%.! s0 is made from a state s 2 S first by nonde-
%terministically selecting an action-distribution pair (a, ��) such that (s, a, ��) 2
%7
%steps, and second by making a probabilistic choice of the target state s0 accord-
%ing to the distribution ��, such that ��(s0) > 0. A path represents a particular
%resolution of both nondeterminism and probability: it is a non-empty finite
%or infinite sequence of probabilistic transitions  = s0
%a0,��0 ...! s1
%a1,��1 ...! �� �� �� such
%that s0 = s. We denote by (i) the (i + 1)-th state of  and last() the
%last state of  if  is finite. An adversary represents a particular resolution
%of nondeterminism only. Formally, an adversary of a MDP is a function A
%mapping every finite path  to a pair (a, ��) such that (last(), a, ��) 2 Steps.
%The evolution of the MDP according to a particular adversary A is a
%measurable set of infinite paths associated with A, which can classically be
%provided with a probability measure. Given a set F  S of target states, let:
%P(s A.!
%
%F) denote the probability of reaching F starting from the initial state
%under A.
%The maximal reachability probability Pmax(s ! F), (resp. minimal reach-
%ability probability Pmin(s ! F)), is the maximum (resp. minimum) proba-
%bility over all the adversaries, with which a given set of states can be reached
%from the initial state: Pmax(s ! F) = maxA{P(s A.!
%
%F)} (resp. Pmin(s !
%F) = minA{P(s A.!
%
%F)}).
%Informally, Pmax(s ! F) (resp. Pmin(s ! F)) represents the probability
%that the MDP reaches F, when all non deterministic choices are as favorable
%(resp. unfavorable) as possible.
%Specification of reachability properties to be checked on a MDP may be
%expressed in PCTL, which is a probabilistic extension of the popular temporal
%logic CTL. In [7] the authors provide an algorithm to enrich the usual algo-
%rithm to check a CTL formula on a finite state system, with the computation
%of maximal and minimal probabilities. This can be done in polynomial time
%by solving linear programming systems.
%PRISM [20] is a probabilistic model checker which provides support for
%analysis of Markov decision processes and performs verification of PCTL for-
%mulae for MDPs, using the model checking algorithm of [7]. The most expen-
%sive part of this is the computation of reachability probabilities. For this there
%are two options, either solution of a linear optimization problem or iterative
%numerical solution techniques (based on dynamic programming). PRISM uses
%the second of these. Each iteration computes new values for the reachability
%probabilities, tending towards the exact solution. The computation is ter-
%minated when it has converged to within the desired precision (parameter "
%specified by the user).
%To analyse an MDP, PRISM has to construct the full reachable state space
%and the transition matrix which represents it. However, the tool can often
%handle very large models because it uses symbolic model checking techniques.
%It uses BDD (binary decision diagram) based data structures, in particular
%MTBDDs (multi-terminal BDDs, see e.g. [15]).
%In this paper, we have made use of a prototype extension of PRISM which
%8
%provides support for analysis of properties based on costs (or conversely, re-
%wards) [22]. Each state or transition of the model can be assigned a cost.
%PRISM allows computation of, for example, the expected amount of cost cu-
%mulated before a certain set of states is reached. Because we use MDPs,
%we must compute either the minimum or maximum expected cost (over all
%resolutions of nondeterminism). In PRISM, these properties are expressed as
%(i) Rmin[true Ugoal ]
%(ii) Rmax[true Ugoal ].
%PRISM uses the algorithms of [3] to perform model checking of these for-
%mulas.

\subsection{Stochastic Two-Player Game}

\section{Protocol Modeling and Verification}

\begin{definition}
An agent \textbf{organisation} is a tuple $O=\langle A, N, R, R_A  \rangle$ where
\noindent
\begin{itemize}
\item $A= \{ a_1,a_2,...,a_n \} $ is a set of agents,
\item $N=\{\{a_i,a_j \}\ :$ ``$a_i$ and $a_j$ are neighbours'' $\}$ is a neighbourhood relation,
\item $R=\{r_1,r_2,\dots,r_k\}$ is a set of resource types,
\item $R_A=\{R_{a_1}, R_{a_2}, \dots, R_{a_n}\}$ is a set of agent resources where $r_j \in R_{a_i} $ $\iff$ ``agent $a_i$ has a resource $r_j$''.
\end{itemize}
\end{definition}

\begin{definition}
A \textbf{task} $T_i=\{r_i\ :\ $``$r_i$ is required by the task $i$''$\}$ is a set of resources that are required to complete it. By $T=\{T_1, T_2,\dots, T_t\}$ we denote a collection of tasks.
\end{definition}

\begin{definition}
A \textbf{team} of agents is denoted by $M_i=\{a_i\ :\ $``$a_i$ is a member of team $i$''$\}$, and the set of all teams is $M=\{M_1, M_2,\dots , M_m\}$. By $\bar{M} = \bigcup_{M_i \in M} M_i$ we denote the set of all agents that are committed to some team. $R_{M_i}=\bigcup_{a \in M_i}R_{a}$ is the set of resources that the team $M_i$ has. The team $M_i$ is said to be able to complete the task $T_j$ iff $T_j \subseteq R_{M_i}$.
\end{definition}

\begin{definition}
Team \textbf{initiation probability} for agent $a$ is the ratio between neighbours that are not committed to any team and total number of neighbours:

\begin{equation}
IP_a = \frac{|\{ a' \in A\ :\ \{a, a'\} \in N \wedge a' \notin \bar{M}  \}|}{|\{ a' \in A\ :\ \{a, a'\} \in N \}|}.
\label{eq:init_prob}
\end{equation}

\end{definition}

\begin{definition}
We define two types of \textbf{reward} for agent $a$. Type $W_1$ rewards agent with 1 point if it is in the team which was able to complete its task after team formation is over or 0 otherwise.Type $W_2$ rewards 1 point to the team which was able to complete its task, and 0 otherwise and the reward is shared equally between team members,

\begin{equation}
W_1(a) = \left\{
  \begin{array}{l l}
    1 & \quad \text{if $\exists M_i . a \in M_i \wedge T_i \subseteq R_{M_i}$,}\\
    0 & \quad \text{otherwise,}\\
  \end{array} \right.
\label{eq:w1agent}
\end{equation}
\begin{equation}
W_2(a) = \left\{
  \begin{array}{l l}
    \frac{1}{|M_i|} & \quad \text{if $\exists M_i . a \in M_i \wedge T_i \subseteq R_{M_i}$,}\\
    0 & \quad \text{otherwise.}\\
  \end{array} \right.
\label{eq:w2agent}
\end{equation}

The rewards of an agent organisation $O$ are defined accordingly as the total reward achieved by its members,
\begin{equation}
 W_1(O) = \sum_{a \in A} W_1(a),
\label{eq:w1organisation}
\end{equation}
\begin{equation}
 W_2(O) = \sum_{a \in A} W_2(a).
\label{eq:w2organisation}
\end{equation}

\end{definition}



\begin{algorithm}[H]
\caption{Offline and online versions of the algorithm}
\label{alg:main_process}
\begin{scriptsize}
\begin{algorithmic}
\Procedure{Offline}{$t$} \Comment{$t$ - number of tasks}
  \State $M = \{M_i = \emptyset\ :\ 1\leq i \leq t\}$ \Comment{initialise empty teams}
  \State $T = \{T_i\neq \emptyset\ :\ T_i \subseteq_{random} R,\ 1\leq i \leq t\}$ \Comment{initialise tasks at random}
  \ForAll {$a \in A$ in random order}
    \State \Call{JoinTeam}{$a$, $T$, $M$}
  \EndFor
  \State perform tasks and compute rewards
\EndProcedure
\Statex
\Procedure{Online}{$t$} \Comment{$t$ - number of tasks}
  \State $M = \{M_i = \emptyset\ :\ 1\leq i \leq t\}$ \Comment{initialise empty teams}
  \ForAll {$a \in A$ in random order}
    \State \Call{JoinTeam}{$a$, $T$, $M$}
  \EndFor
  \State $T = \{T_i\neq \emptyset\ :\ T_i \subseteq_{random} R,\ 1\leq i \leq t\}$ \Comment{initialise tasks at random}
  \State perform tasks and compute rewards
\EndProcedure

\end{algorithmic}
\end{scriptsize}
\end{algorithm}

\begin{algorithm}[H]
\caption{Team joining algorithm (probabilistic and deterministic)}
\label{alg:join_team_org}
\begin{scriptsize}
\begin{algorithmic}
\Procedure{JoinTeam}{$a$, $T$, $M$}
  \ForAll {$T_i \in T$ in random order}
    \If {$a \notin \bar{M}$} \Comment{agent is not committed}
      \If {$|M_i|=0$} \Comment{team for task i is empty}
	\If {$R_a \cap T_i \neq \emptyset$} \Comment{agent has skill}
	  \State with probability $IP_a$: $M_i \leftarrow M_i \cup \{a\}$ \Comment{initiate a team}
	\EndIf
      \ElsIf {$\exists \{a,a'\} \in N . a' \in M_i$} \Comment{there is neighbour in team for task i}
	\If {$R_a \cap T_i \setminus R_{M_i} \neq \emptyset$} \Comment{agent has a missing resource}
	  \State $M_i \leftarrow M_i \cup \{a\}$ \Comment{join team}
	\EndIf
      \EndIf
    \EndIf
  \EndFor
\EndProcedure
\end{algorithmic}
\end{scriptsize}
\end{algorithm}

\begin{algorithm}[H]
\caption{Team joining algorithm (non-deterministic)}
\label{alg:join_team_nondet}
\begin{scriptsize}
\begin{algorithmic}
\Procedure{JoinTeam}{$a$, $T$, $M$}
  \ForAll {$T_i \in T$ in \hl{arbitrary} order}
    \If {$a \notin \bar{M}$} \Comment{agent is not committed}
      \If {$|M_i|=0$} \Comment{team for task i is empty}
	\If {$R_a \cap T_i \neq \emptyset$} \Comment{agent has skill}
	    \State \hl{$M_i \leftarrow M_i \cup \{a\}$  \textbf{or} $M_i \leftarrow M_i$} \Comment{initiate a team or do nothing}
	\EndIf
      \ElsIf {$\exists \{a,a'\} \in N . a' \in M_i$} \Comment{there is neighbour in team for task i}
	\If {$R_a \cap T_i \setminus R_{M_i} \neq \emptyset$} \Comment{agent has a missing resource}
	  \State \hl{$M_i \leftarrow M_i \cup \{a\}$ \textbf{or} $M_i \leftarrow M_i$} \Comment{join a team or do nothing}
	\EndIf
      \EndIf
    \EndIf
  \EndFor
\EndProcedure
\end{algorithmic}
\end{scriptsize}
\end{algorithm}



\subsection{DTMC}
\subsubsection{Protocol}
\subsubsection{Model and Specification}
\subsubsection{Results}

\subsection{MDP}
\subsubsection{Protocol}
\subsubsection{Model and Specification}
\subsubsection{Results}

\subsection{STPG}
\subsubsection{Protocol}
\subsubsection{Model and Specification}
\subsubsection{Results}

\section{Conclusions and Future Work}

We would like to outline several directions for future work:
\begin{itemize}

 \item Conducting experiments with agents having multiple resources, allowing agents to change teams until stable point is reached and check whether it can reached, what is the expected number of steps to reach the stable configuration, etc.

\item Verifying dual properties, i.e. rather than asking ``what is the expected reward for a strategy'', we would like to be able to model-check the properties of the form ``whats is the probability of reaching reward value R with a strategy?''.

 \item In this paper we only considered sequential execution of agent algorithms, in future work we would like to explore parallel execution (i.e. interleaving steps of JoinTeam algorithm for multiple agents), and see what effect this has for agent strategies in both online and offline versions of the algorithm. Also this raises many interesting scheduling problems, as worst and best case scheduling (of action interleaving) scenarios are unrealistic, the challenging question is to both develop realistic schedulers and perform model-checking efficiently.

 \item Another interesting question is construction of optimal agent organisations given the set of tasks and their distributions.This problem is an instance of mechanism design problem in game-theoretic settings, where the designer can have control over neighbourhood structure or agent resources or both.

\end{itemize}


\end{document} 