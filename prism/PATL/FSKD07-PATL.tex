\documentclass[times, 10 pt,twocolumn]{article}
\usepackage{time}
\usepackage{latex8}
\usepackage{amsmath}
\usepackage{amssymb}
\usepackage{stmaryrd}
\usepackage{graphicx}
\usepackage{dsfont}
%\usepackage{fontenc}
%\usepackage{txfonts}

\addtolength{\textheight}{0.65in} \addtolength{\textwidth}{0.15in}

%===========================================environment definition==========================================

\newenvironment{proof}[1][Proof]{\noindent\rm\textit{#1: }}{{\hspace*{\fill}$\square$}\ \newline}
\newtheorem{theorem}{Theorem}{\bfseries}{\rm}
\newtheorem{notation}{Notation}{\bfseries}{\rm}
\newtheorem{definition}[theorem]{Definition}{\bfseries}{\rm}
\newtheorem{corollary}[theorem]{Corollary}{\bfseries}{\itshape}
\newtheorem{lemma}[theorem]{Lemma}{\bfseries}{\itshape}
\newtheorem{example}{Example}{\itshape}{\rm}
\newtheorem{exercise}{Exercise}{\itshape}{\rm}
\newtheorem{note}{Note}{\itshape}{\rm}
\newtheorem{problem}{Problem}{\itshape}{\rm}
\newtheorem{proposition}{Proposition}{\bfseries}{\itshape}
\newtheorem{question}{Question}{\itshape}{\rm}
\newtheorem{remark}{Remark}{\itshape}{\rm}
\newtheorem{solution}{Solution}{\itshape}{\rm}

\newcommand{\bis}{\mathrel{\underline{\leftrightarrow}}}
\newcommand{\mv}[1]{\mathrel{\stackrel{#1}{\rightarrow}}}
\newcommand{\nv}[1]{\mathrel{\stackrel{#1}{\nrightarrow}}}
\newcommand{\nil}{\mathbf{0}}
\newcommand{\nbis}{~{\underline{\leftrightarrow}}\!\!\!\!/~~}

\newcommand{\mb}[1]{\mathbb{#1}}
\newcommand{\mc}[1]{\mathcal{#1}}

\newcommand{\la}{\langle}
\newcommand{\ra}{\rangle}

\newcommand{\lla}{\langle\langle}
\newcommand{\rra}{\rangle\rangle}

\newcommand{\lb}{\llbracket}
\newcommand{\rb}{\rrbracket}

\newcommand{\st}[1]{\stackrel{#1}{\longrightarrow}}
\newcommand{\wt}[1]{\stackrel{#1}{\Longrightarrow}}

\newcommand{\Ppre}{\textsf{Ppre}}

\newcommand{\mydef}{\stackrel{\mathrm{def}}{=}}

%============================================end of header=========================================

%\pagestyle{empty}
\pagestyle{plain}

%-------------------------------------------------------------------------
\begin{document}

\title{Probabilistic Alternating-Time Temporal Logic \\
and Model Checking Algorithm}

\author{Taolue Chen \thanks{The author is partially
supported by Dutch Bsik project BRICKS (Basic Research in
Informatics for Creating the Knowledge Society), 973 Program of
China (2002CB312002), NSF of China (60233010, 60273034, 60403014),
863 Program of China (2005AA113160, 2004AA112090).} \thanks{%Corresponding author.
Email address: chen@cwi.nl.} \\
CWI, Department of Software Engineering, \\ P.O. Box 94079, 1090GB
Amsterdam, The Netherlands %\\
%\and Jian Lu \thanks{The author is partially supported by
%Supported by 973 Program of China (2002CB312002),
%NSF of China (60233010, 60273034, 60403014), 863 Program of China (2005AA113160, 2004AA112090).}  \\
%State Key Laboratory of Novel Software Technology, \\
%Nanjing University, Nanjing, Jiangsu, P.R.China 210093
}

%\author{
%\begin{tabular}{cc}
%Taolue Chen\thanks{The author is partially supported by Project
%BRICKS (Basic Research in Informatics for Creating the Knowledge
%Society).} \thanks{Corresponding author. Email: chen@cwi.nl.} &
%Jian Lu \thanks{The author is partially supported by Supported by
%973 Program of China (2002CB312002), NSF of China (60233010,
%60273034, 60403014), 863 Program of China (2005AA113160,
%2004AA112090).}\\
%CWI, Department of Software Engineering, & State Key Laboratory of
%Novel
%Software Technology, \\
%PO Box 94079, 1090GB Amsterdam, The Netherlands & Nanjing
%University, Nanjing,  China 210093\\
%\end{tabular}
%}

\maketitle \thispagestyle{empty}

% complexity of *
% future work
% restrict hte precision of the timing constraint or the granularity of the semantics
% are concerned with

\begin{abstract}

%\rotatebox[origin=c]{180}{$\varmathbb{E}$}$A$ and remember to


 Last decade witnesses an impressive
development of embedded reactive systems, which motivates the
research of \emph{open} systems, where multiple components
interact with each other and their environment and these
interactions decide the behavior of the system. A natural
``common-denominator" model for open systems is the
\emph{concurrent game structure}, in which several players can
concurrently decide on the behavior of the
system. \emph{Alternating-time temporal logic} (ATL), %due to Alur, Henzinger and Kupferman,
is a temporal logic geared towards the specification and
verification of properties in open systems, which allows to reason
on the existence of strategies for coalitions of players in order
to enforce a given property. \emph{Probabilistic systems}, i.e.
system models in which transitions are equipped with random
information, receive increasingly attention in recent years. In
this paper, we propose to study the \emph{open probabilistic
system}. We extend the framework of ATL in the probabilistic
setting, following the style of \emph{probabilistic computation
tree logic} (PCTL), %due to Hansson and Jonsson,
and obtain two \emph{probabilistic alternating-time temporal
logics}, i.e. PATL and PATL$^*$. They are interpreted over
\emph{probabilistic concurrent game structures}, which is a
probabilistic extension of multi-player concurrent game structure.
We develop model checking algorithms for both PATL and PATL$^*$. A
more expressive logic, \emph{probabilistic game logic}, is also
addressed.
%
%
%It is well recognized that the control problem is closely related
%to (multi-player) games:
%
%Last decade witnesses an impressive development of embedded
%reactive systems, %which often involve multiple concurrent
%components that can interact with each other.
%In the context of
%system verification and control, they are often modelled as
%potential collaborators or adversaries in a multi-player game with
%temporal objectives (e.g. safety).
%which are often modelled as potential collaborators or adversaries
%in a multi-player game with temporal objectives. %Uncertainty, as one of the
%major sources of system complexity,  which %inspires
%involves the research of
%
% is often %naturally
%modelled as probabilistic state changes.
%
 %which require solving a
%two-player concurrent stochastic game.
\end{abstract}
%
%Probabilistic model checking is a technique to verify system
%models in which transitions are equipped with random information.
%Popular models are discrete- and continuous-time Markov chains
%(DTMCs and CTMCs, respectively), and variants thereof which
%exhibit nondeterminism. Efficient model-checking algorithms for
%these models have been developed, have been implemented in a
%variety of software tools, and have been applied to case studies
%from various application areas ranging from randomized distributed
%algorithms, computer systems and security protocols to biological
%systems and quantum computing. The crux of probabilistic model
%checking is to appropriately combine techniques from numerical
%mathematics and operations research with standard reachability
%analysis. In this way, properties such as ``the (maximal)
%probability to reach a set of goal states by avoiding certain
%states is at most 0.6'' can be automatically checked up to a
%user-defined precision. Markovian models comprising millions of
%states can be checked rather fast.
%
\section{Introduction} \label{sec1}

Last decade witnesses an impressive development of embedded
reactive systems, which calls for systematic design and
verification methodologies that can cope with the complexity in
the system design, analysis and implementation. Model checking
\cite{CGP00} is a well-established technique for verifying whether
a system (typically represented by some formal models, say
automata) satisfies a given property. Usually, temporal logics
have been used to specify these properties: \emph{linear time}
temporal logics (typically LTL, \cite{Pnu77}) expresses properties
on each single execution of the model, while \emph{branching time}
temporal logics (typically CTL, \cite{CE81, QS82}) handle the
computation tree of the model. While the computer science
community has concentrated on verification of systems, the control
theory community has always had \emph{synthesis} as their main
objective, as opposed to analysis. This results in a new framework
in the field of verification: control (and controller synthesis
\cite{RW89}), which was developed in the late 80's. Its goal is to
build a controller that should prevent the (model of the) system
from having unwanted behaviors.

In system verification, especially in the area of control, it
turns out to be essential to distinguish between \emph{open} and
\emph{closed} systems \cite{HP85}. For a \emph{closed} system, the
behavior is completely determined by the state of the system while
for an \emph{open} system, the behavior is affected both by its
internal state and by the ongoing interaction with its
environment. Thus, while in a closed system all the
nondeterministic choices are internal, and resolved by the system,
in an open system there are also external nondeterministic
choices, which are resolved by the environment \cite{Hoa85}. In
the literatures, while closed systems are naturally modelled as
Kripke structures (or labelled transition systems), %a general model for open systems,
%which can accommodate a wide variety of notions of composition, is
%the \emph{concurrent game structure}.
a natural ``common-denominator" model for compositions of open
systems is the \emph{concurrent game structure} \cite{AHK02}, in
which several
players concurrently decide on the behavior of the system. %In particular,
%closed systems correspond to the special case of a single player.
In particular, Kripke structures can be viewed as game structures
with a single player which represents the player \emph{system}. In
this case, game structures degenerate to Kriple structures.


%The Kripke structure is a natural ``common-denominator" model for
%closed systems, %independent of whether the high-level description
%%of a system is given, say, as a product of state machines or as a
%%set of guarded commands on variables.
%In analogy, the natural ``common-denominator" model for
%compositions of open systems is the \emph{concurrent game
%structure}. In particular, Kripke structures can be viewed as game
%structures with a single player $sys$, which represents the
%system.


It is well recognized that the control problem is closely related
to (multi-player) games: solving such a game amounts to computing
a strategy (if it does exist) for a player so that she surely
reaches a state where she is declared to be the winner. %In that
%case, the underlying model is \emph{not} only a simple automaton
%(LTS or Kripke structure), but rather a ``concurrent game
%structure" (CSGs) \cite{AHK02}, in which several agents
%concurrently decide on the behavior of the system.
In this case, one often feels convenient to reason with
strategies, for which a new flavor of temporal logics has been
defined: \emph{alternating-time} temporal logic \cite{AHK02}. This
logic distinguishes itself in the sense that it allows to express,
for instance, that a coalition of players has a strategy in order
to always reach a winning location, or to always avoid reaching a
bad location. In \cite{AHK02}, several alternating-time temporal
logics are introduced to specify properties of game structures,
including  the  CTL-like logic ATL, and the CTL$^*$-like logic
ATL$^*$.
%For example, the ATL formula $\lla i\rra\diamondsuit p$ is true at
%state $s$ iff player $i$ can force the game from $s$ into a state
%that satisfies the proposition $p$.

%\paragraph{Probability}
Most of the researchers agree that \emph{uncertainty} is one of
the major sources
of system complexity. %The probabilities can be exploited to model
%a certain probability of error or other stochastic behavior both
%occurring in various real world applications, e.g. certain
%randomized algorithms or communication protocols over faulty
%media.
This motivates the investigation of \emph{probabilistic systems},
where state transition encodes the probability of making a
transition between states rather than just the existence of such
transition. Recently, the specification and verification of the
systems in
this flavor %such kind of systems
have received a lot of attention and are subject of study of a
rapidly growing research community, for which we invite the
readers to \cite{BHHKS04} for a comprehensive exposition. In
particular, from the modelling perspective, besides the basic
stochastic models such as DTMC and CTMC, taking probabilities into
account in addition to nondeterministic behavior also expands the
possibilities of modelling certain aspects of the system under
consideration; From the specification perspective,
\emph{probabilistic computation tree logic} (PCTL) was introduced
by Hansson and Jonsson \cite{HJ94} and concentrates on
probabilistic model checking w.r.t. probabilistic deterministic
systems \cite{HJ94} or nondeterministic systems \cite{BdeA95}.
%
%
%
%
% In
%the literatures, uncertainty in the environment is naturally
%modelled by probabilistic state changes and thus constitutes so
%called \emph{probabilistic systems}.
% the efforts to
%cover the behavior of probabilistic systems have received a lot of
%attentions.



%Performance and dependability evaluation aim at forecasting system
%behavior in a quantitative way by trying to answer questions
%related to the performance and dependability of systems. Typical
%problems that are addressed are: how many clients can this le
%server adequately support, how large should the buffers in a
%router be to guarantee a packet loss of at most 10..6, or how long
%does it take before 2 failures have occurred?

% \paragraph{Our contributions}
In this paper, we devote ourselves to coping with \emph{open
probabilistic system}. In a nutshell, we extend the framework of
ATL to a probabilistic setting, following the style of PCTL, and
obtain two \emph{probabilistic alternating-time temporal logics},
i.e. PATL and PATL$^*$. We propose the semantics over
\emph{probabilistic concurrent game structures}, which is a
probabilistic extension of multi-player concurrent game structure
\cite{AHK02}. Besides these contributions, the main focus of this
paper is the model checking algorithms for both PATL and PATL$^*$,
where the basic idea is to reduce the model checking problem to
the decidability of the theory of \emph{real closed fields}. %It turns out
%that solving a two-player concurrent stochastic game plays a
%central role.
Under condition that \emph{equality constraints} are not admitted
(see below for explanation), we develop two model checking
procedures for both PATL and PATL$^*$. If we do not restrict the
precision of the constraint, %do allow equality constraints,
as for PATL, we still can handle it by some sophistic encodings
while for PATL$^*$, we conjecture it is undecidable. We also
address a even more expressive logic, called \emph{probabilistic
game logic}, trading the lack of decidability of model checking.
Some illuminating discussions are provided regarding these three
logics. We assume %that the readers are
familiarity of basic knowledge on temporal logic, including linear
time and branching time temporal logic, $\omega$-regular languages
and their connections. For the former, the readers are referred to
\cite{CGP00} whilst for the latter, one can consult to, say,
\cite{Tho97}. We also assume that the readers have some basic
notions on complexity theory and game theory, for which we refer
the readers to \cite{Pap94} and \cite{Owe95} respectively.


%\paragraph{Structure of the paper}
This paper is set up as follows: In section \ref{sec2}, we first
present the probabilistic concurrent game structure, %which extends
%the concurrent game
%structure in \cite{AHK02} into probabilistic setting.
then we introduce the probabilistic alternating-time temporal
logic, including its syntax and semantics. In section \ref{sec3},
we present the model checking algorithms for PATL. In section
\ref{sec4}, we propose PATL$^*$ and address its model checking
algorithm, then we sketch a more expressive extension, i.e. the
probabilistic game logic. %how to
%extend the PATL studied in the previous sections to PATL$^*$%,
%which is more expressive, yet remains the decidability. We also
%discussion a even more expressive logic, called probabilistic game
%logic, which loses the decidability of model checking.
We conclude our work in section \ref{sec5} where some related
works are also discussed in brief.

%%%%%%%%%%%%%%%%%%%%%%%%%%%%%%%%%%%%%%%%%%%%%%%%%%%%%%%%%%%%%%%%%%%%%%%%%%%%%%%%%
%\section{Preliminaries}
%%%%%%%%%%%%%%%%%%%%%%%%%%%%%%%%%%%%%%%%%%%%%%%%%%%%%%%%%%%%%%%%%%%%%%%%%%%%%%%%%

\section{Probabilistic Alternating-time Temporal Logic}
\label{sec2}

In this section, we shall introduce the \emph{probabilistic
alternating-time temporal logic} (PATL) and its semantics model,
namely, \emph{probabilistic concurrent game structure} (pCGS).
Before starting our exposition, let us first fix some general
notations. For a countable set $X$, a \emph{probability
distribution} on $X$ is a function $\delta: X\mapsto [0,1]$ such
that $\sum_{x\in X}\delta(x)=1$. We denote the set of probability
distributions on $X$ by $\mc{D}(X)$. For a probability
distribution $\delta\in \mc{D}(X)$ we define $||\delta||$, the
\emph{support} of $\delta$, as $||\delta||=\{x\in X\mid \delta(x)>
0\}$. For all $k\in \mb{N}$, by $[k]$ we denote the set
$\{1,\cdots, k\}$. For a finite set $X$, we denote the set of
\emph{finite words} over $X$ by $X^*$, the set of \emph{infinite
words} ($\omega$-words) over $X$ by $X^{\omega}$, and the set of
nonempty finite words over $X$ by $X^+$. %$\varepsilon$.
We define $X^{\infty}$ as $X^*\cup X^{\omega}$. For $\lambda\in
X^{\infty}$ and $n\in \mb{N}$, we write $\lambda(n)$ for the
$n$-th letter in $\lambda$, $\lambda[n]$ for the \emph{suffix}
starting at $\lambda(n)$, i.e. $\lambda[n]=
\lambda(n),\lambda(n+1), \cdots$. In addition, we write
$|\lambda|$ for the length of $\lambda$ in case $\lambda\in X^*$
and $\lambda(\uparrow)$ for
$\lambda(|\lambda|-1)$, namely, the last element of $\lambda$. %It is well-known that
%a LTL formula can denote a set of infinite words. % and
%$\alpha$$\upharpoonleft_n$ for the prefix of length $n$ of
%$\alpha$. For $\alpha=a_0a_1\cdots$, we denote
%$\alpha(r)\alpha(r+1)\cdots \alpha(s)=a_r\cdots a_s$ by
%$\alpha[r,s]$.

\subsection{Probabilistic Concurrent Game Structure}

\begin{definition}\label{pcgs}[Probabilistic Concurrent Game Structure] A
\emph{probabilistic concurrent game structure} (pCGS) is a tuple
$\mc{G}=\langle k, Q, \Pi, \pi, d, \delta\rangle$ with the
following components:
\begin{itemize}
  \item $k\in \mb{N}$ is the number of \emph{players} (a.k.a. \emph{agents}).
  We identify the players with the numbers in $[k]$;

  \item $Q$ is a finite set of \emph{states};

  \item $\Pi$ is a finite set of \emph{atomic propositions}
  (a.k.a. \emph{observations});

  \item $\pi: Q\mapsto \wp(\Pi)$ is the \emph{labelling function} (a.k.a. \emph{observation mapping}).
  Namely, for each state $q\in Q$, $\pi$ assigns to $q$ a set $\pi(q)\subseteq \Pi$ of
  propositions that hold true at $q$.

  \item For each player $a\in [k]$ and each state $q\in
  Q$, a natural number $d_a(q)\geq 1$ of \emph{moves} available at state
  to $a$. We identify the moves of player $a$ at state $q$
  with the number $1,\cdots, d_a(q)$. For each state $q\in Q$, a
  \emph{move vector} at $q$ is a tuple $\la j_1, \cdots, j_k\ra $ such that
  $1\leq j_a\leq d_a(q)$ for each player $a$. Given a state $q\in
  Q$, we write $D(q)$ for the set $\{1,\cdots, d_1(q)\} \times
  \cdots\times \{1,\cdots, d_k(q)\}$ of move vectors. The function
  $D$ is called \emph{move function}. For each player $a\in [k]$,
  we write $C(a)$ for the set $\bigcup_{q\in Q}[d_a(q)]$ of all of the
  possible moves available for $a$. The function $C$ is called \emph{choice function}.

  \item For each state $p\in Q$ and each move vector $\la j_1, \cdots
  j_k\ra\in D(p)$, a \emph{probabilistic transition function} $\delta$, which gives
  the (conditional) probability $\delta(q \mid p, \la j_1, \cdots, j_k\ra)$ of a transition from
  state $p$ to state $q$ for all $q\in Q$ if each player $a\in [k]$ chooses
  move $j_a$. Note that %for simplicity, sometimes
  we also write this probability as $\delta(p, \la j_1, \cdots, j_k\ra)(q)$, to emphasize that
  $\delta(p, \la j_1, \cdots, j_k\ra)$ is a probability distribution on $Q$.
\end{itemize}
\end{definition}

The number of states of the structure $\mc{G}$ is $n=|Q|$. The
number of transitions of $\mc{G}$ is $m=\sum_{q\in
Q}d_1(q)\times\cdots\times d_k(q)\times n$, namely, $m=|D|\cdot
n$. Note that unlike in Kripke structure, the number of
transitions is not bounded by $n^2$. For a fixed alphabet $\Pi$ of
propositions, the \emph{size} of $\mc{G}$ is $\mc{O}(m)$.

For two states $p$ and $q$, we say that $q$ is a \emph{successor}
of $p$ if there is a move vector $\la j_1, \cdots j_k\ra\in D(p)$
such that $q\in ||\delta(p, \la j_1, \cdots j_k\ra)||$. We write
$p\rightarrow q$ if $q$ is a successor of $p$.

%\vspace{2mm}

The semantics of pCGS, intuitively, is as follows: At each state
$q\in Q$, each player $a\in[k]$ chooses a move $1\leq j_a\leq
d_a(q)$ \emph{simultaneously} and \emph{independently}. The game
then proceeds to the successor state $q'$ with probability
$\delta(q'\mid q, \la j_1, \cdots, j_k\ra)$, for every state
$q'\in Q$. A \emph{path} of $\mc{G}$ from a state $q_0$ is an
infinite sequence $\lambda= q_0, q_1, \cdots$ of states such that
for all positions $i\geq 0$, the state $q_{i+1}$ is a successor of
state $q_i$. We denote by $\Omega$ the set of all paths. Moreover,
we associate the set
\[\Omega_q = \{\lambda= q_0, q_1, \cdots \mid q_0=q \mbox{ and }q_i\rightarrow q_{i+1}\mbox{ for any }i\in \mb{N}\}\]
the paths starting at state $q$. %For a path $\lambda$ and a
%position $i\geq 0$, we use $\lambda(i)$ to denote the $i$-th state
%of $\lambda$.
Given a path $\lambda=q_0,q_1, \cdots$, we associate
$\pi(\lambda)$ with $\lambda$ as $\pi(q_0),\pi(q_1),\cdots$. By
this way, we can use an LTL (linear time temporal logic) formula
$\Psi$ over $\Pi$ to denote a set of paths $\Xi\subseteq \Omega$
by defining $\Xi=\{\lambda\mid \pi(\lambda)\models \Psi\}$.% Since
%this is rather standard in the literatures, we omit further
%detailed expositions.

\vspace{2mm} Some natural model can be recovered from the general
definition of pCGS. In particular, we distinguish the following
special classes.
%of probabilistic concurrent game structures.

\begin{itemize}
  \item  A Markov Decision Process (MDP, \cite{Put94}) is a
  probabilistic concurrent game structure with $k=1$, namely, only
  one player.

  \item A concurrent game structure $\mc{G}$ is \emph{deterministic},
  if for all $q\in Q$ and all move vector $\la j_1, \cdots
  j_k\ra\in D(q)$, there is a $p\in Q$ such that
  $\delta(p \mid q, \la j_1, \cdots, j_k\ra)=1$. %Note that
  In this case,
  a pCGS is %probabilistic concurrent game structure is
  simply %exactly
  the ``concurrent game structure" in \cite{AHK02}.
\end{itemize}

The interesting point with the framework of ATL, compared with
standard temporal logics, is that it allows quantifications on
\emph{strategies} of (coalitions of) players. A \emph{coalition}
$A$ is a subset of the set of players, namely, $A\subseteq [k]$.
We write $\bar{A}$ for $[k]\setminus A$. In the sequel we
introduce the notions of \emph{strategy} and \emph{outcome}, in
order to define the semantics of PATL.

\begin{definition} Assume a probabilistic concurrent game
structure $\mc{G}=\langle k, Q, \Pi, \pi, d, \delta \rangle$.
\begin{itemize}
  \item A \emph{strategy} for player $a\in [k]$ is a mapping
  $\sigma_a: Q^+\mapsto \mc{D}(C(a))$ that associates with every
  nonempty finite sequence $\lambda\in Q^+$ of states,
  representing the past history of the game, a \emph{probability
  distribution} $\sigma_a(\lambda)$ used to select the next move.
  Thus, the choice of the next move can be \emph{history-dependent} and
  \emph{randomized}. The strategy $\sigma_a$ can prescribe only moves that
  are available to player $a$, i.e. for all sequences $\lambda\in
  Q^+$ and $q\in Q$, it is required that $||\sigma_a(\lambda(\uparrow))||\subseteq [d_a(\lambda(\uparrow))]$. %$\sigma_a(\lambda q)(j)>0$
%  iff $1\leq j\leq d_a(q)$.
  We denote by $\Upsilon_a$ the set of all
  strategies for player $a\in [k]$.

  \item Let $A\subseteq [k]$ be a coalition. A \emph{move} for
  $A$ from a state $q$ is a family $(j_a)_{a\in A}$: one move
  for each player in $A$. We write ${\rm \sf Mv}(q, A)$ to represent the set
  of all possible moves for $A$ from $q$. Moreover, a \emph{coalition
  strategy} $\sigma_A$ for $A$ is a family $(\sigma_a)_{a\in A}$, where $\sigma_a\in \Upsilon_a$.
  Given a move $c\in
  {\rm \sf Mv}(q, A)$ and $\bar{c}\in {\rm \sf Mv}(q, \bar{A})$,
  by abuse of notations, we write $\delta(q, c\cdot \bar{c})$ for the probabilistic
  transition function corresponding to these choices. Note that here $c\cdot \bar{c}$ is the
  \emph{move vector} defined in Definition \ref{pcgs}, namely, $c\cdot \bar{c}\in D(q)$.

  \item Let $A\subseteq [k]$ be a coalition and $\sigma_A$ and $\sigma_{\bar{A}}$ be the coalition
  strategies for $A$ and $\bar{A}$ respectively. Once the starting state $q$ and the coalition
  strategies $\sigma_A$ and $\sigma_{\bar{A}}$ for the two coalitions have been chosen,
  the game is reduced to an ordinary \emph{stochastic process} (typically,
  a discrete time Markov chain). Hence, the probabilities of \emph{events} are
  uniquely defined in a standard way, where an event $\mc{E}\subseteq \Omega$ is measurable set of
  paths. For an event $\mc{E}\subseteq \Omega$, we denote the probability that a
  path belongs to $\mc{E}$ when the game starts form $q$ and the
  players follow the strategies $\sigma_A$ and $\sigma_{\bar{A}}$ by
  $\mb{P}_q^{\sigma_A, \sigma_{\bar{A}}}(\mc{E})$. Similarly, for
  a measurable function $f$ that associates a number in $\mb{R}\cup\{\infty\}$
  with each path, we denote by $E_q^{\sigma_A, \sigma_{\bar{A}}}\{f\}$ the expected
  value of $f$ when the game starts from $q$ and the strategies $\sigma_A$ and
  $\sigma_{\bar{A}}$ are followed. As usual, we denote by $\Theta_i$ the \emph{random
  variable} over sample space $\Omega$ representing the $i$-th state of a path; formally,
  $\Theta_i$ is a variable that assumes value $q_i$ on the path
  $q_0, q_1,\cdots$.

  Given a coalition strategy $\sigma_{A}=(\sigma_a)_{a\in A}$, we define the
  set of possible \emph{outcomes} of $\sigma_{A}$ from a
  state $q\in Q$ to be the set ${\rm \sf Outcomes}(q, \sigma_A)$ of \emph{probabilistic measure} %stochastic processes
  that the players in $A$ enforce when they follow the strategy $\sigma_{A}$, namely, for each $a\in A$, player $a$ follows
  strategy $\sigma_a$. We use $\mb{O}_q^{\sigma_A}$ to range
  over ${\rm \sf Outcomes}(q, \sigma_A)$.
\end{itemize}
\end{definition}

% some routine definitions

%%%%%%%%%%%%%%%%%%%%%%%%%%%%%%%%%%%%%%%%%%%%%%%%%%%%%%%%%%%%%%%%%%%%%%%%%%%%%%%%%%%%%%%%%%%%%%%%%%%%

%First, let us fix some general notations. For a finite set $X$ we
%denote the set of finite words over $X$ by $X^*$, the set of
%infinite words ($\omega$-words) over $X$ by $X^{\omega}$, and the
%empty word by $\varepsilon$. We define $X^{\infty}=X^*\cup
%X^{\omega}$. For $\alpha\in X^{\infty}$ and $n\in \mb{N}$, we
%write $\alpha(n)$ for the $n$-th letter in $\alpha$ and
%$\alpha$$\upharpoonleft_n$ for the prefix of length $n$ of
%$\alpha$. For $\alpha=a_0a_1\cdots$, we denote
%$\alpha(r)\alpha(r+1)\cdots \alpha(s)=a_r\cdots a_s$ by
%$\alpha[r,s]$.

%\begin{definition}
%A path of $\mc{G}$ is an infinite sequence $\pi=s_0s_1\cdots$ of
%states of states in $Q$, such that $(s_i, s_{i+1})\in T$ for all
%$i\geq 0$. We denote the set of all paths by $\Omega$.
%\end{definition}

%\begin{definition}[Path]
%A path of $\mc{T}$ is a finite or infinite sequence
%$\pi=s_0s_1\cdots$ of states, such that $(s_i, s_{i+1})\in T$ for
%all $i$.
%
%Given a finite path $\rho=s_0,s_1,\cdots s_n$, the length $|\rho|$
%of $\rho$ is equal to $n$. For finite path $\pi$, the length is
%equal to $\infty$. Given a (finite or infinite) path
%$\pi=s_0,s_1,\cdots$ and $i<|\pi|$, we denote the $i$th state of
%$\pi$ by $\pi[i]$ (i.e. $\pi[i]=s_i$).
%
%Furthermore, we denote by $Path^*$ (resp. $Path^{\omega}$) the set
%of finite (resp. infinite) pathes of a given Markov chain and by
%$Path^*(s)$ (resp. $Path^{\omega}(s)$) the set of finite (resp.
%infinite) pathes of a given Markov chain starting at the state
%$s$.
%
%For any $s\in S$, we associate the set
%\[\Omega_s = \{s_0s_1\cdots\mid s=s_0\mbox{ and }(s_i, s_{i+1}\in T\mbox{ for any }i\in \mb{N}\}\]
%\end{definition}

%\begin{definition}[Basic Cylinder] For $\rho\in Path^*$ the basic
%cylinder of $\rho$ is defined as
%\[\Delta(\rho)=\{\pi\in Path^{\omega}\mid \pi[1..|\rho|]=\rho \}\]
%\end{definition}
%
%\begin{definition}[Probability Space of a Markov Chain] Given a
%Markov chain $M=(S, T, p)$ and $s\in S$, we define a probability
%space
%\[\Psi^s=(\Delta(s), \Delta^s, \mathbb{P}_s)\]
%such that
%\begin{itemize}
%\item $\Delta^s$ is the $\sigma$-algebra generated by the empty
%set and the basic cylinders over $S$ that are contained in
%$\Delta(s)$;
%
%\item $\mathbb{P}_s$ is the uniquely induced probability measure
%which satisfies the following: $\mathbb{P}_s(\Delta(s)=1$ and for
%all basic cylinders $\Delta(s,s_1,\cdots s_n)$ over $S$,
%\[\mathbb{P}_s(\Delta(s,s_1,\cdots s_n))=p_{ss_1}\cdots
%p_{s_{n-1}s_n}\]
%\end{itemize}
%\end{definition}

\subsection{Probabilistic Alternating-time Temporal Logic}
%In this section, we shall introduce \emph{probabilistic
%alternating time temporal logic} (PATL), including its syntax and
%semantics.
The temporal logic PATL is defined w.r.t. a finite set $\Pi$ of
\emph{atomic propositions} and a finite set $[k]$ of
\emph{players}.

\begin{definition} The syntax of \emph{PATL} is defined by the
following grammar:
\begin{eqnarray*}
  \phi_s, \psi_s & ::= & \top \mid P\mid \neg \phi_s\mid
  \phi_s \wedge \psi_s \mid \lla A\rra [\psi_p]_{\bowtie r}\\
  \psi_p & ::= & \mc{X}\phi_s\mid \phi_s\mc{U}\psi_s \mid  \phi_s\mc{R}\psi_s
\end{eqnarray*}
where $P\in \Pi$, $A\subseteq [k]$, $\bowtie \in\{<, \leq, =,
\geq, >\}$ and $r\in [0,1]$.

In addition we also can define some standard abbreviations. %shorthands.
For instance, $\bot$ for $\neg \top$, $\lla A\rra
[\mc{F}\psi_s]_{\bowtie r}$ for $\lla A\rra
[\top\mc{U}\psi_s]_{\bowtie r}$, $\lla A\rra
[\mc{G}\psi_s]_{\bowtie r}$ for $\lla A\rra
[\bot\mc{R}\psi_s]_{\bowtie r}$, etc.
\end{definition}

%As in ATL, the operator $\lla \cdot \rra$ is a \emph{probabilistic
%quantifier}, and
The logic PATL is similar to the probabilistic branching-time
temporal logic PCTL, only that $[\cdot]_{\bowtie r}$ quantifier is
parameterized by sets of players. $\mc{X}$, $\mc{U}$, $\mc{R}$ are
standard \emph{temporal operators}. As usual, the formulae with a
subscript $s$ are \emph{state} formulae while the ones with
subscript $p$ are \emph{path} formulae. We will stick to these
conventions throughout the rest of the paper.

%Note that $\lla \emptyset\rra[\psi_p]_{\bowtie r}$ corresponds to
%the PCTL formula $[\psi_p]_{\bowtie r}$.

%Why we introduce weak until?

Note that contrary to usual definitions of ATL, we include the
``release" modality $\mc{R}$. The reason lies in that in
\cite{LMO07}, it is proved that modality $\lla A\rra\mc{R}$ can
\emph{not} be expressed using only $\lla A\rra\mc{U}$ and $\lla
A\rra\mc{G}$ in ATL \cite{AHK02}. It is not difficult to see that
their arguments hold true in the probabilistic setting. Moreover,
due to the same reason, we include $=$ in the constraint
$[\cdot]_{\bowtie r}$, in contrast to the case in PCTL.
%
\paragraph{Semantics.} PATL formulae are interpreted over states of
pCGS which has the same propositions and players. The labelling of
the states of $Q$ with propositions is used to evaluate the atomic
formulae of PATL. The logical connectives $\neg$ and $\wedge$ have
the standard interpretations. The most interesting case is the
state-formula $\lla A\rra [\psi_p]_{\bowtie r}$. Intuitively, it
holds in a state $q$ of $\mc{G}$ iff there exists a
\emph{coalition strategy} for coalition $A$ in order to enforce
the probability of paths which satisfy the subformula $\psi_p$
along all the outcomes to meat the constraint specified by
$[\cdot]_{\bowtie r}$. To put it concretely, we can consider a
game between a protagonist and an antagonist. The former
represents coalition $A$ and accordingly, the latter represents
coalition $\bar{A}$ and they both follow their own coalition
strategies. The protagonist wins the game if in the resulting
stochastic process, the probability measure of pathes which
satisfy the subformula $\psi_p$, read as a linear temporal formula
whose outermost operator is $\mc{X}$, $\mc{U}$ or $\mc{R}$,
fulfills the constraint $[\cdot]_{\bowtie r}$; otherwise, the
antagonist wins. The PATL formula $[\psi_p]_{\bowtie r}$ is
satisfied at the state $q$ iff the protagonist has a winning
strategy in this game.

\vspace{2mm}  We are now in a position to define the semantics
formally. We write $\mc{G}, q\models \varphi$ to indicate that the
state $q$ satisfies the formula $\varphi$ in the structure
$\mc{G}$. When $\mc{G}$ is clear from the context, we omit it and
write $q\models \varphi$. The \emph{satisfaction relation}
$\models$ is defined for all states $q$ and paths $\lambda$ of
$\mc{G}$, inductively as follows.
%
%In order the heal this, we provide two sorts of semantics, namely,
%the \emph{precise} semantics and \emph{approximate} semantics,
%between which the main difference lies in that in the approximate
%semantics, we take the \emph{value}, which can not be achieved by
%any strategy (since there does not exist any optimal strategies
%generally) while in the precise, we take the real strategies
%(namely, $\epsilon$-optimal strategies) into account. We will
%discuss later under which conditions the two semantics coincide.
%
%Given a state $q\in Q$, a set $A\subseteq [k]$ of players, and a
%coalition strategy $\sigma_A=(\sigma_a)_{a\in A}$, we define the
%outcomes of $F_A$ from $q$ to be the set ${\rm \sf Ourcomes}(q,
%F_A)$ of \emph{stochastic processes} that the players in $A$
%enforce when they follow the strategies in $F_A$.
%
%We can now turn to the formal definition of the semantics of PATL.
%
%\begin{definition}[Semantics]
%
%\begin{eqnarray*}
% q\models true \\
% q\models a  & \Leftrightarrow & a\in L(s)\\
% \lambda \models [\phi]_{\bowtie r} & \Leftrightarrow & \mathbb{P}(\pi\in
%Path^{\omega}(s)\mid \pi\models \phi)\bowtie p
%\end{eqnarray*}
%
%For path formula, the two interpretations are the same.
\[
\begin{tabular}{lll}
 $q\models$    $true$ \\

 $q \models P$  & $\Leftrightarrow$   & $P\in \pi(q)$\\
 $q \models \neg \phi_s$ & $\Leftrightarrow$   & $q\not\models \phi_s$\\
 $q \models \phi_s \wedge \psi_s$  & $\Leftrightarrow$   & $q \models \phi_s$ and $q \models \psi_s$\\

 $q \models \lla A\rra [\psi_p]_{\bowtie r}$ & $\Leftrightarrow$   & there
is a coalition strategy\\
& & $\sigma_A$ such that for any \\
& & $\mathbb{O}_q^{\sigma_A}\in {\rm\sf Outcomes}(q, \sigma_A)$,
\\
& & $\mathbb{O}_q^{\sigma_A}(\{\lambda\in \Omega_q \mid
\lambda\models \psi_p\})\bowtie r$\\

 $\lambda \models \mc{X}\phi_s$ & $\Leftrightarrow$   &
 $\lambda(1)\models \phi_s$\\

 $\lambda \models \phi_s\mc{U}\psi_s$ & $\Leftrightarrow$   &
 $\exists i\in \mb{N}$. $\lambda(i)\models \psi_s$ and for \\
 & & any $0\leq j<i$, $\lambda(j)\models
 \phi_s$\\

 $\lambda \models \phi_s\mc{R}\psi_s$ & $\Leftrightarrow$   &
 %$\exists i\in \mb{N}$. $\lambda_i\models \psi_s$ and for any $0\leq j<i$, $\lambda(j)\models
% \phi_s$\\
 $\lambda \not\models \neg \phi_s\mc{U} \neg \psi_s$\\
\end{tabular}
\]
%\end{definition}

%When a system behaves according to a strategy $\eta$ in the
%evolution from $q_0\in \textsf{Loc}$, and has reached $q_n$
%following the sequence $q_0, \cdots, q_n$, it will choose the
%next-state distribution with probability $Q_{\eta}(i\mid q_0\cdots
%q_n)$.
%
%Therefore, we can associate with each finite sequence $q_0,
%\cdots, q_n$ starting at the root $q=q_0$ of
%
%\[\mu_s^-(\Delta)=\inf_{\sigma} \sup_{\pi}\mu_{\sigma,
%\pi}(\Delta) \quad\quad  \mu_s^+(\Delta)=\sup_{\sigma}
%\inf_{\pi}\mu_{\sigma, \pi}(\Delta) \]


\section{Model Checking PATL} \label{sec3}

In this section, we aim at presenting algorithms for model
checking PATL. As we have seen in the previous section, the
semantics of PATL has a close relationship with two-player
concurrent game, which is %also naturally
in turn the case for our model checking algorithm. To this end,
first we introduce some notions and properties regarding the
two-player concurrent game. Most of materials for this are taken
from \cite{deAM04}.

\subsection{Two-player concurrent game}

A two-player concurrent game structure is a special case of
probabilistic concurrent game structure when $k=2$ (recall that in
this case, the two players are named after 1 and 2). In game
theory, the notion of \emph{winning condition} plays an important
role. Here, let us consider the cast that it is expressed by LTL
formulae. Namely, given an LTL winning condition $\Psi$, by abuse
of notation we denote equally by $\Psi$ the set of paths
($\omega$-regular set) $\lambda\in \Omega$ that satisfy $\Psi$. It
is well known that this set is \emph{measurable} for any choice of
strategies for the two players. Hence, the probability that a path
satisfies $\Psi$ starting from state $q\in Q$ under strategies
$\sigma_1$, $\sigma_2$ of the two players can be denoted by
$\mb{P}_q^{\sigma_1,\sigma_2}(\Psi)$. Given a state $q\in Q$ and a
winning condition $\Psi$, one is often interested in finding the
maximal probability with which player $a\in \{1,2\}$ can ensure
that $\Psi$ holds from $q$. This probability is called the
\emph{value of the game} $\Psi$ at $q$ for player $a\in \{1,2\}$.
This value for player 1 is given by the function $\lla
1\rra\Psi:Q\mapsto [0,1]$, defined as
%
%\[ \lla 1\rra\Psi(q) = \sup_{\sigma_1 \in \Upsilon_1} \inf_{\sigma_{2} \in
%\Upsilon_{2}} \mathbb{P}_q^{\sigma_1, \sigma_{2}}(\{\lambda\in
%\Omega_q \mid \lambda\models \psi\})$ and
%
%min=$\inf_{\sigma_A \in \Upsilon_A} \sup_{\sigma_{\bar{A}} \in
%\Upsilon_{\bar{A}}} \mathbb{P}_q^{\sigma_A,
%\sigma_{\bar{A}}}(\{\lambda\in
% \Omega_q \mid \lambda\models \psi\})$
%
\[ \lla 1\rra\Psi(q) = \sup_{\sigma_1 \in \Upsilon_1} \inf_{\sigma_{2} \in
\Upsilon_{2}} \mathbb{P}_q^{\sigma_1, \sigma_{2}}(\Psi)\] and the
value for player 2 is given by the function $\lla 2\rra\Psi$,
defined symmetrically.

%min=$\inf_{\sigma_A \in \Upsilon_A} \sup_{\sigma_{\bar{A}} \in
%\Upsilon_{\bar{A}}} \mathbb{P}_q^{\sigma_A,
%\sigma_{\bar{A}}}(\{\lambda\in
% \Omega_q \mid \lambda\models \psi\})$

It is also well known that concurrent games enjoy a
\emph{quantitative} version of determinacy \cite{Mar98}, which
implies that for all LTL conditions $\Psi$ and all $q\in Q$, the
following equation holds:
\[\lla 1\rra\Psi(q) + \lla 2\rra\neg \Psi(q) =1 \]
%
A strategy $\sigma_1$ is \emph{optimal} if for all $q\in Q$,
\[\inf_{\sigma_2\in \Upsilon_2}\mathbb{P}_q^{\sigma_1, \sigma_{2}}(\Psi)= \lla 1\rra\Psi(q) \]
A strategy $\sigma_1$ is $\epsilon$-\emph{optimal} if for all
$q\in Q$,
\[\inf_{\sigma_2\in \Upsilon_2}\mathbb{P}_q^{\sigma_1, \sigma_{2}}(\Psi)\geq \lla 1\rra\Psi(q)-\epsilon \]
The same notions for player 2 can be defined symmetrically. We
note that the quantitative determinacy of concurrent games is
equivalent to the existence of $\epsilon$-optimal strategies for
both players for all $\epsilon>0$ at all states $q\in Q$.

\vspace{2mm}

In order to compute the value of the game, the following notion,
namely, \emph{predecessor operator} is essential. Let $\mc{F}$ be
the space of all functions $Q\mapsto [0,1]$ that map states into
the interval $[0,1]$. Given two functions $f,g\in \mc{F}$, we
write $f>g$ (resp. $f\geq g$) if $f(q)>g(q)$ (resp. $f(q)\geq
g(q)$) at all $q\in Q$, and we define $f\wedge g$ and $f\vee g$ by
\begin{eqnarray*}
  (f\wedge g)(q) &=& \min \{f(q), g(q)\}\\
  (f\vee g)(q)   &=& \max \{f(q), g(q)\}
\end{eqnarray*}
for all $q\in Q$. We denote by $\textbf{0}$ and $\textbf{1}$ the
constant functions that map all states into 0 and 1. For all $f\in
\mc{F}$, we denote by $\textbf{1}-f$ the function defined by
$(\textbf{1}-f)(q)=1-f(q)$ for all $q\in Q$. Given a subset
$S\subseteq Q$ of states, we denote by $[S]$ the \emph{indication
function} of $S$, defined by $[S](q)=1$ if $q\in S$ and $[S](q)=0$
o.w.

\vspace{2mm}

The \emph{quantitative predecessor operators} $\Ppre_1$, $\Ppre_2:
\mc{F}\mapsto \mc{F}$ are defined for every $f\in \mc{F}$ by
\[
\Ppre_1(f)=\sup_{\sigma_1\in \Upsilon_1}\inf_{\sigma_2\in
\Upsilon_2}E_s^{\sigma_1, \sigma_2}\{f(\Theta_1)\}
\]
and symmetrically for $\Ppre_2$. Recall that %$\Theta_1$ is the
%random variable representing the $1$-th state of a path; formally,
$\Theta_1$ is a random variable that assumes value $q_1$ on the
path $q_0, q_1,\cdots$. Intuitively, the value $\Ppre_i(f)$ is the
maximum expectation for the next value of $f$ that player $a\in
\{1,2\}$ can achieve. Given $f\in \mc{F}$ and $a\in \{1,2\}$, the
function $\Ppre_a(f)$ can be computed by solving the following
\emph{matrix game} at each state $q\in Q$:
\[
\Ppre_1(f)=val_1[\sum_{p\in Q}f(p)\cdot \delta(p\mid q, j_1,
j_2)]_{1\leq j_1\leq d_1(q), 1\leq j_2\leq d_2(q) }
\]
where $val_1(A)$ denotes the value obtained by player 1 in the
\emph{matrix game} $A$. The existence of solutions to the above
matrix games, and the existence of \emph{optimal randomized
strategies} for players 1 and 2, are guaranteed by the celebrated
minmax theorem \cite{vNM47}. The matrix games may be solved using
traditional linear programming algorithms.

%%%%%%%%%%%%%%%%%%%%%%%%%%%%%%%%%%%%%%%%%%%%%%%%%%%%%%%%%%%%%%%%%%%%%%

\paragraph{Quantitative $\mu$-calculus over game structure}
In order to write the solution of games w.r.t. $\omega$-regular
winning conditions, we introduce the \emph{quantitative game
$\mu$-calculus}. The formulae of the \emph{quantitative game
$\mu$-calculus} are generated by the grammar
\[
\phi::= [S] \mid x \mid \phi\vee\phi \mid \phi\wedge\phi \mid
\Ppre_1(\phi)\mid \Ppre_2(\phi) \mid \mu x.\phi\mid \nu x.\phi
\]
for atomic propositions $[S]$ where $S\subseteq Q$ and variables
$x$ from some fixed set $X$.

The atomic propositions of quantitative $\mu$-calculus formulae
correspond to subsets of states of the probabilistic concurrent
game structure. As usual, a formula $\phi$ is \emph{closed} if
every variable $x$ in $\phi$ occurs in the scope of a fixpoint
quantifier $\mu x$ or $\nu x$.

Let $\mc{E}: X\rightarrow \mc{F}$ be a \emph{variable valuation}
that associates a function $\mc{E}(x)\in \mc{F}$ with each
variable $x\in X$. We write $\mc{E}[x\mapsto f]$ for the valuation
that agrees with $\mc{E}$ on all variables, except that $x\in X$
is mapped to $f\in \mc{F}$. Given a valuation $\mc{E}$, every
formula $\phi$ of quantitative game $\mu$-calculus defines a
function $\lb \phi\rb_{\mc{E}}\in \mc{F}$:
\[
\begin{tabular}{lll}
  $\lb f\rb_{\mc{E}} $ & =  & $f$\\
  $\lb x\rb_{\mc{E}} $ & =  & $\mc{E}(x)$\\
  $\lb \Ppre_1(\phi)\rb_{\mc{E}} $ & =  & $\Ppre_1(\lb\phi\rb_{\mc{E}})$\\
  $\lb \Ppre_2(\phi)\rb_{\mc{E}} $ & =  & $\Ppre_2(\lb\phi\rb_{\mc{E}})$\\
  $\lb \phi_1\{\genfrac{}{}{0pt}{}{\wedge}{\vee}\}\phi_2\rb_{\mc{E}} $ & =  & $\lb \phi_1\rb \{\genfrac{}{}{0pt}{}{\wedge}{\vee}\} \lb \phi_2\rb $\\
  $\lb \{\genfrac{}{}{0pt}{}{\mu}{\nu}\}x.\phi\rb_{\mc{E}} $ & = & $\{\genfrac{}{}{0pt}{}{\sup}{\inf}\}\{f\mid f=\lb\phi\rb_{\mc{E}[x\mapsto f]}\}$
\end{tabular}
\]
%
% semantics
%
The quantitative game $\mu$-calculus defined above suffices for
writing the solution formulae of games with $\omega$-regular
winning conditions, which are, in our case, expressed by LTL
formulae. To put it more concretely, given a concurrent game with
LTL winning condition $\Psi$, one can solve it by providing a
quantitative game $\mu$-calculus formula $\phi$ such that $\lla
1\rra \Psi =
\lb\phi\rb$. We stress that this fact essentially boils down to % entails % gives rise to %entails, The
an algorithm, which is based on the observation that the value of
%a concurrent game Rabin-Chain winning conditions
a quantitative $\mu$-calculus formula can be expressed as an
elementary formula in the theory of \emph{real closed fields}, and
uses a decision procedure for the theory of reals with addition
and multiplication \cite{Tar51}. In order to understand the
algorithm in the sequel, here we include a simple introduction
on this aspect. %only present a
%simple introduction. (include a simple introduction)
%
%\vspace{2mm}

An ordered field $H$ is real-closed if no proper algebraic
extension of $H$ is ordered. We denote by $\mathbf{R}$ the
real-closed field $(\mb{R}, +, \cdot, 0,1, \leq)$ of the reals
with addition and multiplication. An \emph{atomic formula} $a$ is
an expression of the form $p>0$ or $p=0$ where $p$ is a (possibly)
multi-variate polynomial with integer coefficients. An
\emph{elementary formula} is constructed from atomic formulae by
the grammar
\[\phi::= a \mid \neg \phi \mid \phi\wedge \phi\mid \phi\vee \phi \mid \exists x.\phi\mid \forall x.\phi\]
where $a$ is an atomic formula. The semantics of elementary
formula are given in a standard way. It is well-known that the
theory of real-closed fields in the language of ordered fields is
decidable \cite{Tar51}.

A classical observation is that the minmax value can be written as
an elementary formula in the theory of ordered fields. Namely, let
$A=(a_{ij})$ be a matrix with entries in the ordered field $H$.
Then the statement $y=val_1(A)$ can be written as an elementary
formula over $H$. Following from this, we have:
%
%
%%We start with the following classical observation that the minmax
%value can be written as an elementary formula in the theory of
%ordered fields.
%
%\begin{lemma} \label{matrix}
%  Let $A=(a_{ij})$ ba a matrix with entries in the ordered field
%  $H$. Then the statement $y=val_1(A)$ can be written as an
%  elementary formula over $H$.
%\end{lemma}
%
\begin{lemma}[\cite{deAM04}] \label{express}
  Let $\mc{G}$ be a two-player game and $q$ a state in $\mc{G}$.
  Let $\Psi$ be a Rabin chain condition and let $\lla 1\rra \Psi =
  \lb\phi\rb$. Then statement $\vec{y}=\lb\phi\rb$ can be written
  as an elementary formula in the theory of real closed fields.
\end{lemma}
We note that in the lemma above, the vector $\vec{y}$ is indexed
by the states of $\mc{G}$. Thus for each state $q\in Q$, we denote
the $q$-indexed coordinate of $\vec{y}$ as $(\vec{y})_q$.

\subsection{Algorithm}
%
%The complexity of model checking an ATL formula over a CGS has
%been shown to be linear in both the size of the structure and the
%size of the formula \cite{AHK02}.
%
In this section, we present our model-checking algorithms for PATL
over pCGSs. The algorithms share the same basic structure of those
proposed in \cite{CES86} for CTL (see \cite{CGP00} for a leisure
demonstration). In a nutshell, given a \emph{state} formula
$\phi_s$, the algorithm recursively evaluates the truth-values of
the state subformulae $\psi_s$ of $\phi_s$ at all states, starting
from the propositional formulae of $\phi_s$ and following the
recursive definitions of each modality. The whole process will be
gathered up in a global labelling algorithm. Indeed, since PATL
differs from CTL only in the presence of the $\left\langle
{\left\langle A \right\rangle } \right\rangle[\psi_p]_{\bowtie
r}$, we can exploit the same techniques proposed for CTL to deal
with the operators $\wedge, \vee$, etc. In the algorithms below ,
we only need to examine the case corresponding to $\left\langle
{\left\langle A \right\rangle } \right\rangle[\psi_p]_{\bowtie
r}$.

\vspace{2mm}

As expected, the problem of model checking a \emph{multi-player}
concurrent game structure w.r.t. $\left\langle {\left\langle A
\right\rangle } \right\rangle[\psi_p]_{\bowtie r}$ boils down to
the problem of solving a \emph{two-player} concurrent game. To see
this, it suffices to note that any \emph{coalition strategy} can
be decomposed into single strategy of each player in this
coalition. Assume a probabilistic concurrent game structure
$\mc{G}=\langle k, Q, \Pi, \pi, d, \delta\rangle$. The first step
is to define a two-player concurrent game $\mc{H}$ which is played
by $A$ and $\bar{A}$ where $A\subseteq [k]$ and
$\bar{A}=[k]\setminus A$, as
follows. For simplicity, we denote the player $A$ by 1 %use 1 to denote the player $A$
and accordingly, 2 for $\bar{A}$.
%
%\vspace{2mm}
%
%\noindent
\[\mc{H}=\la Q, \Pi, \pi, \Gamma_1, \Gamma_2, \gamma\ra\]
where
\begin{itemize}
  %\item $S$=$Q$;

  \item For each $q\in Q$, $\Gamma_1(q)=\{(j_a)_{a\in A}\mid 1\leq j_a\leq d_a(q)\}$;

  \item For each $q\in Q$, $\Gamma_2(q)=\{(j_a)_{a\in \bar{A}}\mid 1\leq j_a\leq
  d_a(q)\}$; and

  \item For each $q\in Q$, any $c_1\in \Gamma_1(q)$ and $c_2\in
  \Gamma_2(q)$, $\gamma(p\mid q, c_1, c_2)=\delta(q,
  c_1\cdot  c_2)(p)$ for any $p\in Q$.
\end{itemize}

As stated before, the rest of this subsection will be devoted to
demonstrating how to model check $\lla A \rra[\psi_p]_{\bowtie
r}$. We distinguish two cases, depending on $\bowtie$ (recall that
$\bowtie \in \{<,\leq, =, \geq, >\}$), which are presented in the
following two subsections respectively.

\subsubsection{Case $\bowtie \in \{<,\leq,
\geq, >\}$} \label{nonequal}

It turns out that in this case, we can exploit the results of
two-player game for our model checking algorithm in a somehow
direct way. However, due to the intricacy of the concurrent game
structure, we have to cope with the optimal strategies and
$\epsilon$-optimal strategies more carefully. %it is a bit confusing to give the semantics of PATL.
The following facts are well-known: concurrent games with
reachability winning condition have \emph{memoryless
$\epsilon$-optimal} strategies. However, there are deterministic
concurrent games without \emph{optimal} strategies. This means in
such a game, player 1 can obtain the value \emph{arbitrarily
close} to the \emph{optimal value}, but can \emph{not} achieve the
optimal one, which leads to the following Lemma \ref{cru}.

Given a \emph{path} formula $\psi_p$, we assume that we have
already computed (recursively) the satisfaction sets of all
maximal \emph{state} PATL-subformulae $\gamma_1, \cdots, \gamma_n$
of $\psi_p$, so we can view them as atomic propositions. In
addition we have labelled the states of $\mc{G}$ appropriately
with new atomic propositions $r_1, \cdots, r_n$ (to save
notations, we still denote it by $\mc{G}$).
%Let $\phi_s'$ be the formula we obtain from $\phi_s$ by rewriting
%all subformulae involving
%
Let
\[ \widehat{\psi_p} = \psi_p\{ \gamma_1 \leftarrow r_1, \cdots,  \gamma_n \leftarrow r_n \}\]
be the formula we obtain from $\psi_p$ by replacing each
occurrence of $\gamma_i$ by $r_i$ for $1\leq i\leq n$. %Then
%$\phi_s'$ is a LTL formula over the atomic propositions $r_1,
%\cdots, r_n$ and for all infinite paths $\lambda$ of given system,
%it holds that $\lambda\models \psi_p$ iff $\lambda\models
%\psi'_p$.

\begin{lemma} \label{cru} Given probabilistic concurrent
game structure $\mc{G}=\langle k, Q, \Pi, \pi, d, \delta\rangle$
and PATL formula $\phi_s=\lla A\rra[\psi_p]_{\bowtie r}$, where
$A\subseteq [k]$. The following properties hold:
\begin{itemize}
  \item[(i)] Assume $\bowtie\in \{\geq, > \}$. Then $\mc{G}\models
  \phi_s$ if and only if $\lla 1\rra \widehat{\psi_p} > r$.

  \item[(ii)] Assume $\bowtie\in \{\leq, < \}$. Then $\mc{G}\models
  \phi_s$ if and only if $\lla 2\rra \neg\widehat{\psi_p} < r$.
\end{itemize}
\end{lemma}

Here, we note that  $\neg\widehat{\psi_p}$ can be rewritten as a
path formula, by pushing the negation inwards jumping $\mc{X}$ and
interchanging $\mc{U}$ and $\mc{R}$. The intuition underlying this
lemma lies in that for model checking, we only need to consider
\emph{strict inequality}, since generally, we can only achieve
$\epsilon$-optimal strategies. The correctness of this lemma
follows from the semantics of PATL, the definitions of the optimal
values and $\epsilon$-optimal values and the determinacy theorem
mentioned before. We omit the details because of the space
restriction.
%
%\begin{proof}
%  By definition.
%\end{proof}
%
In addition, due to the duality of $\leq, <$ and $\geq, >$, from
now on we only consider the case for $\geq, >$ (i.e. case (i) in
Lemma \ref{cru}), and another case can be obtained by switching
the roles of two players 1 and 2.
%%%%%%%%%%%%%%%%%%%%%%%%%%%%%%%%%%%%%%%%%%%%%%%%%%%%%%%%%%%%%%%%%%%%%%%%%%%%%%%%%%%%%%%%%%
%
%We reduce the model checking
%
%\begin{theorem} The following properties hold:
%
%\begin{itemize}
%  \item Assume $\bowtie \in \{\geq, >\}$, then $q \models \lla A\rra ([\psi]_{\bowtie r})$
%  iff $\sup_{\sigma_A \in \Upsilon_A} \inf_{\sigma_{\bar{A}} \in \Upsilon_{\bar{A}}} \mathbb{P}_q^{\sigma_A, \sigma_{\bar{A}}}(\{\lambda\in
%  \Omega_q \mid \lambda\models \psi\})\bowtie r$.
%
%  \item Assume $\bowtie \in \{\leq, <\}$, then  $q \models \lla A\rra ([\psi]_{\bowtie r})$
%  iff $\inf_{\sigma_A \in \Upsilon_A} \sup_{\sigma_{\bar{A}} \in \Upsilon_{\bar{A}}}  \mathbb{P}_q^{\sigma_A, \sigma_{\bar{A}}}(\{\lambda\in
%  \Omega_q \mid \lambda\models \psi\})\bowtie r$.
%\end{itemize}
%\end{theorem}
%\vspace{2mm}

In the sequel, we demonstrate how to deal with three temporal
modalities, namely $\mc{X}$, $\mc{U}$ and $\mc{R}$, separately.

\paragraph{Modality $\mc{X}$} First we consider the case
of formula $\lla A\rra [\mc{X}\phi_s]_{\bowtie r}$. Let
$T\subseteq Q$ be defined as $T= \{q\in Q\mid q\models \phi_s \}$.
Clearly,
\begin{equation}  \label{eqn11}
\lla A\rra[\mc{X}\phi_s]_{\bowtie r}= \lb \Ppre_A([T])\rb
\end{equation}
Note that since $\Ppre_A([T])$ is a quantitative $\mu$-calculus
formula, by Lemma \ref{express} the statement $\vec{y}=\lb
\Ppre_A([T]) \rb$ can be written as an elementary formula in the
theory of real closed fields. It follows from Lemma \ref{cru} that
$q\models \lla A\rra[\mc{X}\phi_s]_{\bowtie r}$ (where $\bowtie
\in\{\geq, >\}$) can be encoded as the formula in $(\mb{R}, +,
\cdot, \leq)$, as follows
\[\exists \vec{y}. (\vec{y}=\lb\Ppre_A([T])\rb \wedge (\vec{y})_q \bowtie
r)\]

\paragraph{Modality $\mc{U}$} In this case we consider the case
of formula $\lla A\rra [\phi_s\mc{U}\psi_s]_{\bowtie r}$. As in
the previous case, all the same %we have For simplicity, we denote the value
%$\sup_{\sigma_A \in \Upsilon_A} \inf_{\sigma_{\bar{A}} \in
%\Upsilon_{\bar{A}}} \mathbb{P}_q^{\sigma_A,
%\sigma_{\bar{A}}}(\{\lambda\in
%  \Omega_q \mid \lambda\models \phi_s\mc{U}\psi_s\})$ by $\lla A\rra
%  (\phi_s\mc{U}\psi_s)$.
let $S, T\subseteq Q$ be defined as $S= \{q\in Q\mid q\models
\phi_s\}$ and $T= \{q\in Q\mid q\models \psi_s\}$. Then it is not
difficult to show that
  \[
  \lla A\rra [\phi_s\mc{U}\psi_s]_{\bowtie r}=\lb \mu x. [T]\vee ([S]\wedge \Ppre_A(x) ) \rb
  \]
Note that since $\mu x. [T]\vee ([S]\wedge \Ppre_A(x) ) $ is a
quantitative $\mu$-calculus formula, by Lemma \ref{express} the
statement $\vec{y}=\lb \mu x. [T]\vee ([S]\wedge \Ppre_A(x) ) \rb$
can be written as an elementary formula in the theory of real
closed fields. It follows from Lemma \ref{cru} that $q\models \lla
A\rra[\phi_s\mc{U}\psi_s]_{\bowtie r}$ (where $\bowtie \{\geq,
>\}$) can be encoded as the formula in $(\mb{R}, +, \cdot, \leq)$,
as follows
\begin{equation} \label{eqn2}
\exists \vec{y} (\vec{y}=\lb\mu x. [T]\vee ([S]\wedge \Ppre_A(x) )
\rb \wedge (\vec{y})_q \bowtie r)\end{equation}

\paragraph{Modality $\mc{R}$} In this case we consider the case
of formula $\lla A\rra [\phi_s\mc{R}\psi_s]_{\bowtie r}$. As in
the previous case, all the same let $S, T\subseteq Q$ be defined
as $S= \{q\in Q\mid q\models \phi_s\}$ and $T= \{q\in Q\mid
q\models \psi_s\}$. Then it is not difficult to show that
%\begin{eqnarray*}
% &  &\lla A\rra ([\phi_s\mc{R}\psi_s]_{\bowtie r})\\
% &  &\  =\lb (\nu x. [T]\wedge \Ppre_A(x)) \vee (\mu x. [S]\vee ([T]\wedge \Ppre_A(x)
%  ))\rb
%\end{eqnarray*}
\[
\begin{array}{l}
  \lla A\rra [\phi_s\mc{R}\psi_s]_{\bowtie r} \\
  \quad =\lb (\nu x. [T]\wedge \Ppre_A(x)) \vee (\mu x. [S]\vee ([T]\wedge \Ppre_A(x)
  ))\rb \\
\end{array}
\]
For brevity, let $\chi=(\nu x. [T]\wedge \Ppre_A(x)) \vee (\mu x.
[S]\vee ([T]\wedge \Ppre_A(x) )) $.  Note that since $\chi$ is a
quantitative $\mu$-calculus formula, by Lemma \ref{express} the
statement $\vec{y}=\lb \chi\rb$ can be written as an elementary
formula in the theory of real closed fields. It follows from Lemma
\ref{cru} that $q\models \lla A\rra[\phi_s\mc{U}\psi_s]_{\bowtie
r}$ can be encoded as the formula in $(\mb{R}, +, \cdot, \leq)$ as
\begin{equation}
\label{eqn3} \exists \vec{y} (\vec{y}=\lb\chi\rb \wedge
(\vec{y})_q \bowtie r)
\end{equation}

To conclude, we note (\ref{eqn11})(\ref{eqn2}) and (\ref{eqn3})
are all formulae in the theory of real closed fields and thus they
are \emph{decidable}, which constitute the model checking
algorithm.

\subsubsection{Case $\bowtie\in \{=\}$} \label{equalsection}

The general idea of the algorithm for this case is the same as in
the previous subsection, namely, to reduce the model checking
problem to the decision problem of theory of \emph{real closed
fields}. However, here we follow a rather different reduction,
since we have to cope with equality rather than inequality and
thus only finding the value of two-player concurrent game is not
sufficient for our purpose.

Assume we are checking formula $\lla A\rra [\psi_p]_{= r}$ w.r.t.
a state $q_0\in Q$. What we intent to encode is, intuitively,
\begin{equation} \label{eqn1}
\exists \sigma_A. \forall \sigma_{\bar{A}}.
\mb{P}^{\sigma_A,\sigma_{\bar{A}}}_q (\{\lambda\in
\Omega_q\mid\lambda \models \psi_p\}) = r
\end{equation}
%
Generally, it is not possible to do this. However, thanks to the
simplicity of the \emph{path} formula $\psi_p$ in our setting,
which only involves the modalities $\mc{X}$, $\mc{U}$ and $\mc{R}$
and thus denotes the $\omega$-regular set in $\Sigma_1^0 \cup
\Pi_1^0$ (safety and reachability, which lie in the lowest level
of Borel hierarchy), it is not difficult to see that
\emph{memoryless} strategies suffice for coalition $A$ to win the
game. This is stated in the following lemma, for which the proof
is omitted due to space restriction.
\begin{lemma}\label{memory}
  Memoryless (randomized) strategies are sufficient to win the
  concurrent game with winning conditions expressed by $\omega$-regular set
  which is in $\Sigma_1^0 \cup \Pi_1^0$ with exact probability.
  %safety or reachability
\end{lemma}
%
%\begin{proof}
%  Omitted.
%\end{proof}

%%%%%%%%%%%%%%%%%%%%%%%%%%%%%%%%%%%%%%%%%%%%%%%%%%%%%%%%%%%%%%%%%%%%%%%%%%%%%%%%%%%%%%%%%%%%%%%%%%%%%%%%

%\vspace{2mm} \noindent{\bf \emph{Step} 1}:
\noindent In order to instantiate (\ref{eqn1}), for each state
$q\in Q$ and action $c\in \Gamma_1(q)$, we introduce a first-order
variable $X_{q,c}$. Intuitively, $X_{q,c}$ denotes the
\emph{probability} of choosing the action $c$ in the state $q\in
Q$, thus it represents the strategy of player 1 (i.e., $A$).
Correspondingly, for each state $q\in Q$ and action $d\in
\Gamma_2(q)$, we also introduce a first-order variable $Y_{q,d}$
for player 2. %For brevity, we $\Delta(X)$ to denote
%\[
%\bigwedge_{X_{q,c}}(0\leq X_{q,c}\leq 1)\wedge
%    \bigwedge_{q\in Q}(\sum_{c\in \Gamma_1(q)} X_{q,c}= 1)
%\]
Then (\ref{eqn1}) turns out to be
%
\[%\left.
\begin{array}{l}
  \exists \{X_{q,c}\mid q\in Q, c\in \Gamma_1(q)\}\\
  \quad \bigwedge_{X_{q,c}}(0\leq X_{q,c}\leq 1)\wedge
    \bigwedge_{q\in Q}(\sum_{c\in \Gamma_1(q)} X_{q,c}= 1)\\
  \forall \{Y_{q,d}\mid q\in Q, d\in \Gamma_2(q)\} \\
  \quad \bigwedge_{Y_{q,d}}(0\leq Y_{q,d}\leq 1)\wedge
    \bigwedge_{q\in Q}(\sum_{d\in \Gamma_2(q)} Y_{q,d}= 1)\\
 \vspace{2mm}
  \qquad \wedge\ \mb{P}^{\sigma_A,\sigma_{\bar{A}}}_q (\{\lambda\in
\Omega_{q_0}\mid\lambda \models \psi_p\}) = r
\end{array}
\]
It remains to encode $\mb{P}^{\sigma_1,\sigma_2}_q (\{\lambda\in
\Omega_{q_0}\mid\lambda \models \psi_p\}) = r$. To this end, we
proceed by a case analysis on the form of $\psi_p$.
%
%\item
%\vspace{2mm} \noindent {\bf \emph{Step} 2}:
%We then encode the maximal and minimal probability w.r.t. player
%2.
%%Clearly, the remaining thing is to construct the formula
%$W_{s,\psi}$, which is the most difficult task. %As we mentioned
%before, this will be done in an inductive way according to the
%structure of $\psi$.
\begin{enumerate}
\item $\psi_p=\bigcirc \phi_s$. First note that here
$\phi_s$ is a state formula. %%Then by Theorem \ref{md}(1), $W_{s,\psi} \mydef
%%      \mb{P}^+_{s}(\bigcirc\psi_1) \bowtie r$.
%According to Definition \ref{maxmin},
It is easy to see that the probability equals
\[ \sum_{c\in \Gamma_1(q)}\sum_{d\in \Gamma_2(q)}
      \sum_{q'\in Q, q' \models \psi_s} X_{q,c} \cdot Y_{q,d} \cdot \gamma(q,c,d)(q')\]
It follows that (\ref{eqn1}) can be instantiated further as
\begin{equation}\label{eqn4}
\begin{array}{l}
  \exists \{X_{q,c}\mid q\in Q, c\in \Gamma_1(q)\}\\
  \quad \bigwedge_{X_{q,c}}(0\leq X_{q,c}\leq 1)\wedge
    \bigwedge_{q\in Q}(\sum_{c\in \Gamma_1(q)} X_{q,c}= 1)\\
  \forall \{Y_{q,d}\mid q\in Q, d\in \Gamma_2(q)\} \\
  \quad \bigwedge_{Y_{q,d}}(0\leq Y_{q,d}\leq 1)\wedge
    \bigwedge_{q\in Q}(\sum_{d\in \Gamma_2(q)} Y_{q,d}= 1)\\
 \vspace{2mm}
  \qquad \wedge\ \sum_{c\in \Gamma_1(q_0)}\sum_{d\in \Gamma_2(q_0)}\\
  \qquad\qquad \sum_{q'\in Q, q' \models \psi_s} X_{q,c} \cdot Y_{q,d} \cdot \gamma(q,c,d)(q') = r
\end{array}
\end{equation}
%
%%%%%%%%%%%%%%%%%%%%%%%%%%%%%%%%%%%%%%%%%%%%%%%%%%%%%%%%%%%%%%%%%%%%%%%%%%%%%%%%%%%%%%%%%%%%%%%%%%%%%%%%%%%%%%%%%
%\vspace{2mm}
%
\item $\psi_p=\psi_1\mc{U}\psi_2$. As in the previous case, here
both $\psi_1$ and $\psi_2$ are {\em state} formulae. This case is
much more involved than the previous one. However, the basic idea
remains the same. Recall that the main task is to compute the
probability $ \mb{P}^{\sigma_1,\sigma_2}_{q_0} (\{\lambda\in
\Omega_{q_0} \mid\lambda \models \psi_1\mc{U}\psi_2 \})$ for which
we have to consult to the well known Chapman-Kolmogorov equation, as follows. %According
%to Definition \ref{maxmin}, it is easy to see that
\[
\begin{array}{l}
\mb{P}_s(\psi_p)= \\
\quad  \left\{\begin{tabular}{ll}
                                    $1$&if $s\models \psi_2$\\
                                    $0$&if $s\not\models \psi_1$\\
                                    $\sum_{c\in \Gamma_1(q)}\sum_{d\in \Gamma_2(q)}
                                    \sum_{q'\in Q}$\\
                                    \quad $X_{q,c} \cdot Y_{q,d} \cdot
                                    \gamma(q,c,d)(q') \cdot \mb{P}_{q'}(\psi_p)$&o.w.\\
                                 \end{tabular}
                          \right.
\end{array}
\]
%However, different from the previous case, now it is difficult to
%reduce this problem to solving
%      a pure linear programming, since in the definition, the case distinctions have to be involved. Fortunately,
%      we can still borrow the same idea, because actually what we need  is to express the linear inequation
%      in $(\mb{R},+,\cdot,\leq)$ rather than to find the
%      solution concretely. By this observation, we set
%      \[\begin{tabular}{ll}
%        min & $z_s$\\
%        s.t. & for any $a\in \mc{I}(s)$,\\
%             &  $\left\{\begin{tabular}{ll}
%                    $z_s = 1$ & if $s\models \psi_2$\\
%                    $z_s = 0$ & if $s\not\models\psi_1$\\
%                    $z_s = \sum_{c\in \Gamma_1(q)}\sum_{d\in \Gamma_2(q)}
%                    \sum_{t\in S}$ & \\
%                    \quad $X_{q,c} \cdot Y_{q,d} \cdot
%                    \gamma(s,c,d)(t)\cdot z_t$ & o.w.\\
%
%
%
%                    \sum_{t\in S\setminus S_0}\mb{P}(s,a,t) z_t$ & \\
%                    $+\sum_{t\in S_0}\mb{P}(s,a,t)Y_{t,\psi}$
%                 \end{tabular}
%           \right.$  \\
%        \end{tabular}
%      \]
\noindent For each $s\in Q$, we introduce a variable $Z_s$. And
for simplicity, we define $\Im(Z_s, Z_t)$ as
\[
\begin{array}{l}
  \Im(Z_s, Z_t) \mydef \\
  \quad (s\models \psi_2 \Rightarrow Z_s=1) \wedge (s\not\models \psi_1 \Rightarrow Z_s=0) \wedge\\
  \quad (s\not\models\psi_2 \wedge s\models\psi_1\\
  \qquad \Rightarrow Z_s= \sum_{c\in \Gamma_1(q)}\sum_{d\in \Gamma_2(q)} \sum_{t\in Q} \\
  \qquad \qquad\qquad\qquad  X_{q,c} \cdot Y_{q,d} \cdot \gamma(s,c,d)(t)\cdot Z_t )%$ & o.w.\\
\end{array}
\]

It follows that that (\ref{eqn1}) can be instantiated as %$W_{s,\psi}$ is defined as
%\begin{eqnarray*}
%      &&\exists \{Z_s\mid s\in Q\}. (0\leq Z_s\leq 1)\wedge \Im(Z_s, Z_t) \wedge (Z_s = r) \\
%      %&&\quad \wedge (\forall \{Z'_s\mid S\setminus S_0\}. (0\leq Z'_s\leq 1)\wedge \Im(Z_s, Z_t) \Rightarrow (Z'_s\geq
%Z_s))
%    \end{eqnarray*}
%
\begin{equation}\label{eqn5}
\exists \{Z_q\mid q\in Q\}. (0\leq Z_q\leq 1)\wedge
\bigwedge_{p,q\in Q} \Im(Z_p, Z_q) \wedge (Z_{q_0} = r)
\end{equation}
%
\item $\psi_p=\psi_1\mc{R}\psi_2$. This case is almost the same as
the previous one and is left to the readers to work out the
details.
\end{enumerate}
To conclude, since (\ref{eqn4}) and (\ref{eqn5}) are formulae in
the theory of real closed fields and thus they are
\emph{decidable}, we are done.

%

\paragraph{Complexity} It is easy to see that for the
algorithm of each modality, the resulting formula in order field
language is \emph{polynomial} in the size of a given pCGS and PATL
formula (we take the standard measure on the size of the formula).
In addition, the number of quantifier alternations is
\emph{fixed}. Thus we obtain the E{\small XPTIME} upper complexity
bound for each oracle query following \cite{Tar51}\cite{Bas99}. So
the overall complexity of our algorithm is in ${\rm P}^{\rm
EXPTIME}$, which is in turn in E{\small XPTIME}.
%\begin{remark}
%  \begin{itemize}
%    \item we have to use randomized strategy while not only
%    deterministic strategy for ATL. However, the memory is not
%    needed;
%
%    \item the complexity of our algorithm is EXPTIME.
%  \end{itemize}
%\end{remark}

%%%%%%%%%%%%%%%%%%%%%%%%%%%%%%%%%%%%%%%%%%%%%%%%%%%%%%%%%%%%%%%%%%%%%%%%%%%%%%%%%%%%%%%%%%%%%%%%%%%%%%%%%
\subsection{Alternative Decision Procedure} \label{approach2}
In the last subsection, we consider both the cases that
$\bowtie\in \{<, \leq, =, \geq, >\}$. In this subsection, we
present yet another algorithm, inspired by a recent result on
solving two-player concurrent game. However, this only applies to
the case that strict equality $=$ is \emph{not} allowed in the
constraint.

Recall that following the constructions in Section \ref{nonequal},
we can reduce the problem of model checking PATL formula
$\phi_s=\lla A\rra[\psi_p]_{\bowtie r}$ to the problem of deciding
$\lla 1\rra \widehat{\psi_p} > r$ in a two-player game, in the
case of $\bowtie\in \{\geq, >\}$ (for $\bowtie\in \{<, \leq\}$,
this can be done in a symmetric way). Thus, to seek an efficient
way to solve two-player game is essential. Fortunately, the
following result actually provides an \emph{oracle} for solving
two-player concurrent game, which is more efficient, compared with
the oracle in the previous subsections.

\begin{theorem}[\cite{CdeAH06}] \label{oracle}
  For all concurrent game structure $\mc{G}$, for all parity objective
  $\Omega_e$ and $\Omega_o$, and for all rationals
  $\varepsilon>0$, for all rationals $r$, whether $\lla 1\rra(\Omega_e)(s)\in
  [r-\varepsilon, r+\varepsilon]$ can be decided in ${\rm NP}\cap \mbox{\rm
  coNP}$.
\end{theorem}
We note that obviously, here $\widehat{\psi_p}$ denotes an
$\omega$-regular property in $\Sigma^0_1\cup \Pi^0_1$ and thus is
a parity objective. Let $\Omega_o$ be the $\omega$-regular set
corresponding to $\psi_p$. By this theorem, our algorithm is
sketched as follows:

\begin{itemize}
  \item[(i)] Query the oracle whether $\lla 1\rra(\Omega_e)(s)\in
             [r-2\varepsilon, r]$? If the answer is ``yes", then the
             algorithm returns ``no";

  \item[(ii)] Otherwise, query the oracle whether $\lla
1\rra(\Omega_e)(s)\in [r, r+2\varepsilon]$? If the answer is
``yes", then algorithm returns ``yes", otherwise, it returns
``no".
\end{itemize}

\paragraph{Complexity} By Theorem \label{oracle},
for each query, the complexity is in ${\rm NP}\cap \mbox{\rm
coNP}$, it follows that the overall complexity of the this
algorithm is in ${\rm P}^{{\rm NP}\cap {\rm coNP}}$, which is a
lower complexity bound than E{\small XPTIME} \footnote{Of course,
we adopt the widely-accepted hypothesis that P$\neq$ NP and the
polynomial hierarchy does not collapse.}.

%%%%%%%%%%%%%%%%%%%%%%%%%%%%%%%%%%%%%%%%%%%%%%%%%%%%%%%%%%%%%%%%%%%%%%%%%%%%%%%%%%%%%%%%%%%%%%%%%%%%%%%%%%%%%

\section{Beyond PATL} \label{sec4}

In this section, we suggest two more formalisms for the
specification of open probabilistic systems. One is PATL$^*$,
which is very natural in the sense that it is akin to PCTL$^*$,
and thus the extension is in analogy with CTL$^*$ w.r.t. CTL. The
other is called \emph{probabilistic game logic} (PGL), which
introduces the \emph{strategy} quantifier. Both of these two
extensions are more expressive than PATL by definition. However,
as we shall demonstrate, the model checking problem for PATL$^*$
(excluding $=$ in the constraint) is still decidable, although in
a higher complexity while the PGL is even more intractable in the
sense that the model checking becomes undecidable. Consequently,
from the perspective of formal verification, PGL is not a good
candidate.

\subsection{PATL$^*$ and Model Checking Algorithm}

The logic PATL is a fragment of a more expressive logic called
PATL$^*$, where the \emph{state} formulae are kept unchanged while
the \emph{path} formulae turn to
\begin{eqnarray*}
  \phi_p, \psi_p & ::= & \phi_s\mid \neg \phi_p\mid
  \phi_p \wedge \psi_p \mid \mc{X}\phi_p\mid \phi_p\mc{U}\psi_p \mid
  \phi_p\mc{R}\psi_p
\end{eqnarray*}
where $P\in \Pi$, $A\subseteq [k]$, $\bowtie \in\{<, \leq, =,
\geq, >\}$ and $r\in [0,1]$.

As for the semantics, in the case of state formulae, the
definition is unchanged w.r.t. PATL, while the path formulae are
interpreted as follows:
\[
\begin{tabular}{lll}
 $\lambda \models \phi_s$ & $\Leftrightarrow$   &
 $\lambda(0) \models \phi_s$\\

 $\lambda \models \neg\phi_p$ & $\Leftrightarrow$   &
 $\lambda \not\models \phi_p$\\

 $\lambda \models \phi_p \wedge \psi_p$ & $\Leftrightarrow$   &
 $\lambda \models \phi_p \wedge \lambda \models \psi_p$\\

 $\lambda \models \mc{X}\phi_p$ & $\Leftrightarrow$   &
 $\lambda[1]\models \phi_p$\\

 $\lambda \models \phi_p\mc{U}\psi_p$ & $\Leftrightarrow$   &
 $\exists i\in \mb{N}$. $\lambda[i]\models \psi_p$ and for \\ & & any $0\leq j<i$, $\lambda[j]\models
 \phi_p$\\

 $\lambda \models \phi_p\mc{R}\psi_p$ & $\Leftrightarrow$   &
 %$\exists i\in \mb{N}$. $\lambda_i\models \psi_s$ and for any $0\leq j<i$, $\lambda(j)\models
% \phi_s$\\
 $\lambda \models \neg (\neg \phi_p\mc{U} \neg \psi_p)$\\
\end{tabular}
\]
%
\paragraph{Model Checking Algorithm} %The model checking algorithm is
%extended to PATL$^*$.
In this section, we explain how to model check a pCGS %against
w.r.t. a PATL$^*$ formula. We emphasize that here, we exclude the
strict equality $=$ in the PATL$^*$ formulae, namely, the results
here can be read as the counterparts of the ones in Section
\ref{nonequal}. We leave the more general case as an open problem.

As before, let $\phi_s$ be a PATL$^*$ state formula. We only need
to consider $\phi_s$ to be of the form $\phi_s=\lla A\rra
[\psi_p]_{\bowtie r}$ since the other cases are either trivial or
can be reduced to this one.
%
%%%%%%%%%%%%% chapter, pp 165 %%%%%%%%%%%%%%%%%%%%%%%%%%%%%%%%%%%%%%
%
Again, we assume that we have already computed (recursively) the
satisfaction sets of all maximal state PATL$^*$-subformulae
$\gamma_1, \cdots, \gamma_n$ of $\psi_p$, so we can view them as
atomic propositions. And we have labelled the states of $\mc{G}$
appropriately with new atomic propositions $r_1, \cdots, r_n$.
%Let $\phi_s'$ be the formula we obtain from $\phi_s$ by rewriting
%all subformulae involving
%
%Let
%\[\psi_p'=\psi_p\{r_1:=\gamma_1, \cdots, r_n:=\gamma_n\}\]
%be the formula we obtain from $\phi_s$ by replacing each
%occurrence of $\gamma_i$ by $r_i$ for $1\leq i\leq n$.
Define $\widehat{\psi_p}$ as before. However, note that in this
case $\widehat{\psi_p}$ is a general LTL formula over the atomic
propositions $r_1, \cdots, r_n$ and for all infinite paths
$\lambda$ of given system, it holds that $\lambda\models \psi_p$
iff $\lambda\models \widehat{\psi_p}$.

Following the construction of Section \ref{nonequal}, we shall
compute $\lla 1\rra \widehat{\psi_p}$. This can be done by the
standard $\omega$-automaton approach: It suffices to encode the
LTL as a Rabin-chain automaton, coping with then the game
structure consisting of the synchronous product of the original
game structure with the Rabin-chain automaton. An
$\omega$-automaton accepts infinite words over a give alphabet. In
the probabilistic setting, a kind of \emph{determinism} is
necessary. %Here we use deterministic parity automata.

\begin{definition} \label{rbc} [Deterministic Rabin chain automata]
A deterministic Rabin chain automaton is a tuple
\[\mc{A}=(\Sigma, S, s_0, \rho, \alpha)\]
where
\begin{itemize}
  \item $\Sigma$ is a finite alphabet;

  \item $S$ is a finite set of states;

  \item $s_0$ is an initial state;

  \item $\rho:S\times\Sigma\rightarrow S$ is a transition
  function; and

  \item $\alpha=\bigvee_{i=0}^{l-1}(\square\diamondsuit U_{2i}\wedge \neg\square\diamondsuit U_{2i+1})$ %$\alpha: S\rightarrow \{0,\cdots, l\}$ is coloring
%  function.
\end{itemize}
where $l>0$ and $\emptyset=U_{2l}\subseteq U_{2l-1}\subseteq
U_{2l-2}\subseteq\cdots\subseteq U_0=S$.
\end{definition}

Given a deterministic Rabin chain automaton $\mc{A}$ and an
infinite word $w\in \Sigma^{\omega}$ over $\Sigma$, we call the
unique sequence $r=q_0,q_1, \cdots$ with $q_{i+1}=\rho(q_i,
w_{i+1})$ for $i\geq 0$ the run of $\mc{A}$ over $w$. For a run
$r=s_0s_1\cdots$, let
\[\lim(r)=\{s\in S\mid s=s_i\mbox{ for infinitely many }i\mbox{'s}\}\]
A run $r$ is \emph{accepting} according to the Rabin chain
condition if there exists some $0\leq i\leq l-1$ such that
$\lim(r)\cap U_{2i}=\emptyset$ and $\lim(r)\cap
U_{2i+1}\neq\emptyset$. We say $\mc{A}$ accepts $w$ if and only if
$r$ is accepting.

The language of a deterministic Rabin chain automaton $\mc{A}$ is
the set
\[\mc{L}_{\omega}(\mc{A})=\{w\in \Sigma^{\omega}\mid \mc{A}\mbox{ accepts }w\}\]

%\end{definition}
The following result is standard. We note that \cite{Pit06}
provides the most efficient construction, to the best of our
knowledge.
\begin{lemma}
  Given an LTL formula $\varphi$ over $\Pi$, a deterministic Rabin
  chain automaton $\mc{A}$ with the alphabet $\Sigma=\wp(\Pi)$ can be
  constructed such that
  \[\mc{L}_{\omega}(\varphi)=\mc{L}_{\omega}(\mc{A})\]
\end{lemma}

Now we are in a position to present the model checking algorithm.
As the first step, we construct a deterministic Rabin chain
automata $\mc{A}=(\wp(\Pi), S, s_0, \rho, \alpha)$ with
$\mc{L}_{\omega}(\widehat{\psi_p})=\mc{L}_{\omega}(\mc{A})$. Then
given $\mc{G}=(k, Q\times S, \Pi, \pi, d, \delta)$, we build a
product pCGS $\mc{G}'=\mc{G}\times \mc{A}$ for $\mc{G}$ and
$\mc{A}$ as follows:
\begin{definition}
$\mc{G}' = (k, Q\times S, \Pi, \pi', d', \delta')$, where
\begin{itemize}
\item $\pi'((q,s))=\pi(q)$;

\item for any $a\in [k]$, $d'_a((q, s))=d_a(q)$;

\item let $\vec{c}= (c_a)_{a\in [k]}$ such that $1\leq c_a\leq
d'_a((q, s))$ for each $a$,
\[\delta'((q,s), \vec{c})((q', t)) = \left\{
\begin{tabular}{ll}
$\delta(q, \vec{c})(q')$ & if $t=\rho(s,\pi(q))$ \\
$0$                      & o.w.;
\end{tabular}
\right.\]
\end{itemize}
\end{definition}

\begin{remark}
  Here, the condition of ``deterministic" is essential to ensure
  the soundness of the construction.
\end{remark}

It is easy to see that there is a one-to-one correspondence
between the pathes of $\mc{G}$ and those of $\mc{G}\times \mc{A}$
by path-lifting. In addition, taking a path $\lambda$ of $\mc{G}$,
lifting it to a path $\lambda'$ of $\mc{G}\times\mc{A}$ and then
reducing it to its second component, gives then run of $\mc{A}$
over $\pi(\lambda)$. Therefore $\mc{G}\times \mc{A}$ also keeps
track whether $\pi(\lambda)$ will be accepted by $\mc{A}$ or not,
namely, whether $\lambda$ satisfies $\psi_p'$ or not.

Define
\[\alpha'=\bigvee_{i=0}^{l-1}(\square\diamondsuit U'_{2i}\wedge \neg\square\diamondsuit
U'_{2i+1})\] where $U'_i=U_i\times S$.

Taking this into account it is easy to show that %for any two
\[
\sup_{\sigma_A \in \Upsilon_A} \inf_{\sigma_{\bar{A}} \in
\Upsilon_{\bar{A}}} \mathbb{P}_q^{\sigma_A,
\sigma_{\bar{A}}}(\{\lambda\in \Omega_q \mid \lambda\models
\psi_p'\})=\]
\[
\sup_{\sigma'_A \in \Upsilon'_A} \inf_{\sigma'_{\bar{A}} \in
\Upsilon'_{\bar{A}}} \mathbb{P}_q^{\sigma'_A,
\sigma'_{\bar{A}}}(\{\lambda'\in \Omega_{q\times s_0} \mid
\lambda\models \alpha'\})\]

%\vspace{2mm}

By this construction, we can turn the \emph{external} winning
condition which is specified by an LTL formula
($\widehat{\psi_p}$) to \emph{internal} winning condition on the
concurrent game structure.  Then we can apply the same techniques
exploited in previous section for PATL. The main difference lies
in that here we have to deal with more involved winning condition.
The following theorem is taken from \cite{deAM04}
\[
\lla 1\rra \alpha' = \lb \eta_{2l-1}x_{2l-1}\cdots \mu x_1\nu
x_0(\bigvee_{i=0}^{2l-1}([C_i]\wedge \Ppre_1(x_i)))\rb
\]
where $\eta_n=\nu$ if $n$ is even and $\eta_n=\mu$ if $n$ is odd;
and $C_i=U_i\setminus U_{i+1}$ for $0\leq i\leq 2l-1$.

By Lemma \ref{express}, clearly, this can be encoded in the theory
of real closed fields. We have done.
\begin{remark}
  We discuss the difficulty of adopting $=$ in the specification briefly. In the case
  of PATL, the formula $\widehat{\psi_p}$ is only a strict subset of LTL,
  for which the memoryless strategies suffice to win the game, as
  we state in Lemma \ref{memory}; However, in the case of
  PATL$^*$, $\widehat{\psi_p}$ is considerably more involved, and
  thus we have to use Rabin chain (parity) condition to encode it.
  The consequence is that generally winning strategies need infinite
  memory, namely, Lemma \ref{memory} fails, which, in turn invalidates the approach
  in Section \ref{equalsection}. We conjecture that in this case, the model checking
  is undecidable. For more discussions we refer the readers to Remark
  \ref{mark3}.
\end{remark}

\paragraph{Complexity of the algorithm} As the case in PATL, we first analyze
the complexity of algorithm of each modality. Since from LTL to
Rabin chain automata, there is a 2E{\small XPTIME} blow up. It
follows that the resulting formula is 2E{\small XPTIME} in the
size of a given probabilistic concurrent game structure and
PATL$^*$ formula with fixed number of quantifier alternations.
Thus we obtain the 3E{\small XPTIME} upper complexity bound
following \cite{Tar51}\cite{Bas99}. So the overall complexity of
our algorithm is in ${\rm P}^{\rm 3EXPTIME}$, which is in turn in
3E{\small XPTIME}.

\subsection{Probabilistic Game Logic}
In this section, we %offer
propose yet another extension, namely \emph{probabilistic game
logic} (PGL), of which even PATL$^*$ is a fragment. To understand
this, let us note that in PATL, the parameterized quantifier $\lla
A\rra$ first stipulates the \emph{existence} of strategies for the
players in $A$, and then \emph{universally} quantifies over the
outcomes of the stipulated strategies. One may generalize it by
separating the two concerns into \emph{strategy} quantifiers and
\emph{path} quantifiers, which leads to the following definition:

\begin{definition} The syntax of \emph{PGL} is defined by the
following grammar:
\begin{eqnarray*}
  \phi_s    & ::= & \top\mid P \mid \neg \phi_s \mid  \phi_s \wedge \phi_s \mid \boxplus A. \theta_t\\
  \theta_t  & ::= & \phi_s \mid \neg\theta_t \mid \theta_t \wedge \theta_t \mid [\psi_p]_{\bowtie r}\\
  \psi_p    & ::= & \theta_t\mid \neg\psi_p \mid \psi_p\wedge \psi_p \mid \mc{X}\phi_p\mid \phi_p\mc{U}\psi_p \mid
  \phi_p\mc{R}\psi_p
\end{eqnarray*}
where $P\in \Pi$, $A\subseteq [k]$, $\bowtie \in\{<, \leq, \geq,
>\}$ and $r\in [0,1]$.
%
%We also define the usual short hands, such as $\bot \stackrel{\rm
%def}{=} \neg \top$,
\end{definition}

We note that in PGL, the constraint $=$ is redundant, since
$\boxplus A. [\psi_p]_{= r}$ can be written as $\boxplus A.
([\psi_p]_{\geq r}\wedge [\psi_p]_{\leq r})$, which is allowed in
PGL, but not in PATL or PATL$^*$.

We now define the semantics of PGL w.r.t. pCGS $\mc{G}$. Recall
that $A\subseteq [k]$ is a coalition and $\sigma_A$ and
$\sigma_{\bar{A}}$ be the coalition strategies for $A$ and
$\bar{A}$ respectively. Once the starting state $q$ and the
coalition strategies $\sigma_A$ and $\sigma_{\bar{A}}$ for the two
coalitions have been chosen, we can obtain a \emph{probabilistic
execution tree}, which is, an unwinding of a (possibly infinite
state) \emph{Discrete Time Markov Chain} (DTMC). We interpret the
\emph{tree formulae} on DTMC (or equivalently, on probabilistic
execution tree), which follows similar way as in PCTL. Formally,
given a coalition strategy $\sigma_{A}=(\sigma_a)_{a\in A}$, we
define the set of possible \emph{probabilistic execution trees} of
$\sigma_{A}$ from a state $q\in Q$ to be the set ${\rm \sf PET}(q,
\sigma_A)$ of unwinding DTMCs that the players in $A$ enforce when
they follow the strategy $\sigma_{A}$. We use
$\mb{T}_q^{\sigma_A}$ to range  over ${\rm \sf PET}(q, \sigma_A)$.
%
%
%
%
%Hence, the probabilities of \emph{events} are
%  uniquely defined in a standard way, where an event $\mc{E}\subseteq \Omega$ is measurable set of
%  paths. For an event $\mc{E}\subseteq \Omega$, we denote the probability that a
%  path belongs to $\mc{E}$ when the game starts form $q$ and the
%  players follow the strategies $\sigma_A$ and $\sigma_{\bar{A}}$ by
%  $\mb{P}_q^{\sigma_A, \sigma_{\bar{A}}}(\mc{E})$. Similarly, for
%  a measurable function $f$ that associates a number in $\mb{R}\cup\{\infty\}$
%  with each path, we denote by $E_q^{\sigma_A, \sigma_{\bar{A}}}\{f\}$ the expected
%  value of $f$ when the game starts from $q$ and the strategies $\sigma_A$ and
%  $\sigma_{\bar{A}}$ are followed. As usual, we denote by $\Theta_i$ the random
%  variable representing the $i$-th state of a path; formally,
%  $\Theta_i$ is a variable that assumes value $q_i$ on the path
%  $q_0, q_1,\cdots$.
Except for $\boxplus A. \theta_t$, the interpretation of
\emph{state} formulae, \emph{tree} formulae and \emph{path}
formulae is defined in the same way as in PATL and PATL$^*$. The
only interesting case is
\begin{eqnarray*}
q \models \boxplus A. \theta_t & \Leftrightarrow \mbox{there is a
coalition strategy }\sigma_A\mbox{ such} \\
& \mbox{that for any } \mathbb{T}_q^{\sigma_A}\in {\rm\sf PET}(q,
\sigma_A), \mathbb{T}_q^{\sigma_A} \models \theta_t
\end{eqnarray*}
We note in the definition above, $\mathbb{T}_q^{\sigma_A}$ is
essentially a DTMC.
%\[
%\begin{tabular}{lll}
% $q\models$    $true$ \\
%
% $q \models P $  & $\Leftrightarrow$   & $P\in \pi(q)$\\
% $q \models \neg \phi_s$ & $\Leftrightarrow$   & $q\not\models \phi_s$\\
% $q \models \phi_s \wedge \psi_s$  & $\Leftrightarrow$   & $q \models \phi_s$ and $q \models \psi_s$\\
%
% $ \lla A\rra ([\psi_p]_{\bowtie r})$ &
%\\
%& &
%\\
% $\lambda \models \mc{X}\phi_s$ & $\Leftrightarrow$   &
% $\lambda(1)\models \phi_s$\\
%
% $\lambda \models \phi_s\mc{U}\psi_s$ & $\Leftrightarrow$   &
% $\exists i\in \mb{N}$. $\lambda_i\models \psi_s$ and for \\ & & any $0\leq j<i$, $\lambda(j)\models
% \phi_s$\\
%
% $\lambda \models \phi_s\mc{R}\psi_s$ & $\Leftrightarrow$   &
% %$\exists i\in \mb{N}$. $\lambda_i\models \psi_s$ and for any $0\leq j<i$, $\lambda(j)\models
%% \phi_s$\\
% $\lambda \models \neg (\neg \phi_s\mc{U} \neg \psi_s)$\\
%\end{tabular}
%\]

%\begin{remark}
%  = is .
%\end{remark}

In \cite{BBFK06}, the authors show, among others, that the
existence of winning strategies in MDPs (a.k.a $1\frac{1}{2}$
games) with \emph{PCTL objectives} is highly undecidable
($\Sigma_1^1$ hard). Since as we state before, MDP is only a
special case of pCGS, by a straightforward reduction, we show this
negative result implies that:

\begin{theorem} \label{un}
  Model checking PGL is undecidable.
\end{theorem}

\begin{remark} \label{mark3}

%\paragraph{Warring}
%To prevent possible misunderstanding, we note the in a recent LICS
%paper by ..., the authors show that ... However, the result
%presented in that paper does not apply directly to our setting.

We note that for PATL, \emph{memoryless} (but randomized)
strategies suffice for coalition of players to achieve the desired
value of the game, thus force the game to favor them; While for
PATL$^*$, both the \emph{memory} and the \emph{randomness} are
required, although in a \emph{finite} fashion. The difference lies
in the \emph{path} formulae admitted in the two logics. From the
$\omega$-regular property perspective, in PATL, the
$\omega$-regular sets specified by path formulae are in
$\Sigma_1^0 \cup\Pi_1^0$ while in PATL$^*$, they need to be
specified by parity condition, and thus beyond this low Borel
hierarchy (although still in $\Sigma_3^0 \cap\Pi_3^0$).
Furthermore, in probabilistic game logic, the winning conditions
are interpreted on \emph{trees} rather than \emph{paths}, which
lifts the \emph{linear} time notion to \emph{branching} time one.
We emphasize that this is precisely the source of the
undecidability, as Theorem \ref{un} indicates. In
this sense, the logic PATL %isolates a subset of properties that ..., Indeed,
identifies a class of properties for open probabilistic systems
for which it suffices to solve iterated finite game in a rather
efficient way, which might be a strong support for it to be a good
candidate
of specification. %, which can be done in linear time.
\end{remark}

%see Kuperfman's remak. (concur'99, robust satisfaction).

%While the outcome of the games that correspond to alternating
%temporal logic are computations, in

\section{Conclusion and Related Works} \label{sec5}

%State transition encode the probability of making a transition
%between states rather than just the existence of such transition.

In our points of view, there are (at least) two main trends in the
research of system verification: (1) from closed systems to open
systems and (2) from the qualitative analysis to quantitative
analysis, which are motivated by the state-of-art of current
embedded reactive systems. To meet these challenges, %progresses,
in this paper, we aim at investigating \emph{open probabilistic
system}: By analyzing the \emph{open} system, one can cope with
not only traditional verification but also control and synthesis
challenges; By introducing probabilistic extension, one can deal
with systems that are subject to various phenomena of stochastic
nature, which is, in our point of view, more akin to the reality
and covers one of the most significant aspects of quantitative
analysis. To instantiate our goal, we extend the framework of
alternating-time temporal logic to a probabilistic setting,
following the style of classical probabilistic computation tree
logic, and thus obtain two \emph{probabilistic alternating-time
temporal logics}, i.e. PATL and PATL$^*$. They can be used to
specify the desire properties of open probabilistic systems. We
also propose probabilistic concurrent game structures as the
general model for such systems, which, are also natural semantics
models for the proposed logics. Based on them, we develop model
checking algorithms for PATL and PATL$^*$. Moreover, some more
expressive logic, which, unfortunately lacks the decidability of
model checking, is also discussed. We believe the works presented
here establish the first (albeit preliminary) step of a general
formal framework to model, specify, analyze, design and even
program open probabilistic systems, in particular, embedded
reactive systems.
%Instead, systems are subject to various phenomena of stochastic
%nature, correctness thus is of a less absolute nature.
%accordingly, instead of checking whether systems failures are
%impossible a more realistic aim is to establish, for instance,
%whether the chance of shutdown occurring is at most 1.10%.
%\begin{itemize}
%  \item alternating-time temporal logic
%
%  \item timed extension
%
%  \item MDP model checking
%\end{itemize}
%In computer system design, we distinguish between \emph{closed}
%and \emph{open} system. A \emph{closed system} is a system whose
%behavior is completely determined by the state of the system. An
%\emph{open system} is a system that interacts with its environment
%and whose behavior depends on this interaction.
\paragraph{Related Works}
There have been extensive works on ATL and its extensions. In
\cite{AHK02}, ATL has been proposed and defined over CGS. In
\cite{HRS02, LMO07}, expressiveness issues are considered for ATL
and ATL$^*$. Complexity results about model checking (for ATL,
ATL$^+$, ATL$^*$) can be found in, among others, \cite{AHK02,
Sch04, LMO07}. % and complexity of satisfiability is addressed in \cite{GD06}.
In \cite{SF06}, the decidability of satisfiability for
alternating-time $\mu$-calculus is shown. In \cite{SSM06}, the
authors present a sound and complete proof system for proving
ATL$^*$ properties over infinite-state systems. The
abstraction-refinement technique is also considered in, among
others, \cite{BK06}. There are also some extensions of the
framework of ATL to quantitative models, in particular, concerning
on real-time aspect, we are aware of two recent works: \cite{HP06}
and \cite{LMO06}, where the former address TATL, which adds freeze
quantifiers to ATL while the latter considers the durational
concurrent game structure and extends ATL following the TCTL
style. Model checking algorithms are given in their own framework
respectively. However, for the probabilistic extension of ATL, to
the best of our knowledge, we are not aware of any other works.

On the other hand, there is a sea of works addressing the formal
verification and control of probabilistic systems. Here, we only
mention the tip of the iceberg and we restrict ourselves to the
%pick up a tiny part of them, namely,
discrete time setting. The probabilistic computation tree logic is
introduced in \cite{HJ94}, where model checking algorithm coping
with DTMC is also provided. \cite{BdeA95} extends this work to
MDP, where the algorithms for MDP w.r.t. PCTL and PCTL$^*$ are
proposed. These algorithm can be seen as the special cases of our
algorithms, sice both the MDP and PCTL are (very) special cases of
pCGS and PATL respectively. As for the control problem, as far as
we know, \cite{BGLBC04} first discusses the controller synthesis
for PCTL, and \cite{BBFK06} further shows that the problem is,
generally undecidable. Our works on probabilistic model checking
might include \cite{CY95}. Research on (stochastic) game is
another line of related works, for which, we only refer to
\cite{Wal04} for a survey and
to its numerous references there. %two
%representative papers which are most related on our work, namely,
%\cite{CdeAH06, deAM04} and numerous references there.

\vspace{2mm}

%\paragraph{Acknowledgement}
\noindent{\bf Acknowledgement} We are grateful to Wan Fokkink and
Tingting Han for their careful reading of a draft of this paper
and valuable comments. We also thank Orna Kupferman for
stimulating discussions at CONCUR'06.

%\begin{itemize}
%
%
%  \item expressive power: LSV paper, \cite{HRS02}.
%
%  %\item satisfiability and axiomatization: CSL'06, \cite{GD06}
%
%  %\item ICTAC'06: Proving ATL* Properties of Infinite-State Systems
%
%  \item traditional model: complexity, see LSV paper ...
%\end{itemize}
%
%\paragraph{Future Works}
%
%The are a lot of works needed to be done. probabilistic game
%logic.
%
%%%%%%%%%%%%%%%%%%%%%%%%%%%%%%%%%%%%%%%%%%%%%%%%%%%%%%%%%%%%%%%%%%%%%%%%%%%%%%%%%%%%%%%%%%%%%%%
\begin{thebibliography}{00}

\bibitem{AHK02} R. Alur, T. A. Henzinger and O. Kupferman. Alternating-time temporal logic.
{\em Journal of ACM} 49(5): 672-713, 2002.

\bibitem{Bas99} S. Basu. New results on quantifier elimination over real closed fields and
applications to constraint databases. {\em Journal of ACM} 46(4):
537-555, 1999.

\bibitem{BBFK06} T. Brazdil, V. Brozek, V. Forejt, and A. Kucera. Stochastic
games with branching-time winning objectives. In Proc. LICS 2006,
pp. 349-358, IEEE Computer Society, 2006.

\bibitem{BdeA95} A. Bianco and L. de Alfaro. Model checking of probabilistic and
nondeterministic systems.  In Proc. FSTTCS 1995. LNCS 1026, pp.
499-513, Springer, 1995.

\bibitem{BGLBC04} C. Baier, M. Gr\"{o}\ss er, M. Leucker, B. Bollig,
and F. Ciesinski. Controller synthesis for probabilistic systems.
In Proc. IFIP TCS 2004. Kluwer, 2004.

\bibitem{BHHKS04} %Christel Baier, Boudewijn R. Haverkort, Holger Hermanns,
%Joost-Pieter Katoen, Markus Siegle:
C. Baier, B. R. Haverkort, H. Hermanns, J. -P. Katoen and M.
Siegle. Validation of Stochastic Systems - A Guide to Current
Research. LNCS 2925, Springer, 2004.

%\bibitem{BK98}C. Baier and M. Kwiatkowska. Model checking for a probabilistic branching time logic with fairness.
%\emph{Distributed Computing}, 11(3):125-155, 1998.

\bibitem{BK06} %Thomas Ball, Orna Kupferman:
T. Ball, and O. Kupferman. An abstraction-refinement framework for
multi-agent systems. In Proc. LICS 2006, pp. 379-388, IEEE
Computer Society, 2006.

\bibitem{CE81} E. Clarke, and E. Emerson. Design and synthesis of synchronization skeletons
using branching-time temporal logic. In {\em Proceeding Logic of
Programs 1981}, LNCS 131, pp.\ 52-71, Springer, 1981.

\bibitem{CGP00} E. Clarke, O. grumberg, and D. Peled. {\rm Model
Checking}. MIT Press, 1999.

\bibitem{CY95} C. Courcoubetis and M. Yannakakis. The compleixity
of probabilistic verification. \emph{Journal of ACM},
42(4):857-907, 1995.

\bibitem{deAM04} L. de Alfaro, and R. Majumdar. Quantitative solution of omega-regular games.
{\em Journal of  Computer and System Science}. 68(2): 374-397,
(2004).

\bibitem{CdeAH06} K. Chatterjee, L. de Alfaro, and T. Henzinger.
The complexity of quantitative concurrent parity games. In Proc.
SODA 2006, pp. 678-687, 2006.

%\bibitem{GD06} V. Goranko, and G. van Drimmelen. Complete axiomatization and decidability of
%Alternating-time temporal logic. {\em Theoretical Computer
%Science}. 353(1-3): 93-117, 2006.

\bibitem{Hoa85} C. Hoare. \emph{Communicating Sequential
Processes}. Prentics-Hall, 1985.

\bibitem{HP85} D. Harel and A. Pnueli. On the development of
reactive systems. In \emph{Logics and Models of Concurrent
Systems}, volume F-13 of \emph{NATO Advanced Summer Institutes},
477-498. Springer, 1985.

\bibitem{HP06} T. A. Henzinger and V. S. Prabhu. Timed Alternating-Time
Temporal Logic. In Proc. FORMATS 2006, LNCS 4202, pp. 1-17, 2006


\bibitem{HRS02} % Aidan Harding, Mark Ryan, Pierre-Yves Schobbens.
A. Harding, M. Ryan, and P. -Y. Schobbens. Approximating ATL$^*$
in ATL. In Proc. VMCAI 2002, LNCS 2294, pp. 289-301, 2002.

\bibitem{LMO06} F. Laroussinie, N. Markey, and G. Oreiby.
Model checking timed ATL for durational concurrent game
strucutures. In Proc. FORMATS 2006, LNCS 4202, pp. 245-259, 2006.

\bibitem{LMO07} %Markey, Nicolas; Oreiby, Ghassan; Laroussinie, Francois:
F. Laroussinie, N. Markey, and G. Oreiby. On the Expressiveness
and Complexity of ATL. In Proc. FoSSaCS 2007, to appear.




%
%\bibitem{BPR96} S. Basu, R. Pollack and M. Roy. On the
%combinatorial and algebraic complexity of quantifier elimination.
%\emph{Journal of ACM}, 43(6):1002-1045, 1996.


%
\bibitem{CES86} E. M. Clarke, E. A. Emerson and A. P. Sistla. Automatic
verification of finite-State concurrent systems using temporal
logic specifications. \emph{ACM Transaction of  Program Language
and System}, 8(2): 244-263, 1986.
%
%%\bibitem{CHL06} T. Chen, T. Han and J. Lu. On the Markovian
%%randomized strategy of controller for Markov decision processes.
%%Technical report, CWI, Amsterdam, The Netherlands, 2006.
%
%\bibitem{CJH03} K. Chatterjee, M. Jurdzinski and T. Henzinger.
%Simple stochastic parity games. In \emph{Proceeding of CSL'03}.
%LNCS 2803, 100-113, Springer, 2003.
%
%\bibitem{CJH04} K. Chatterjee, M. Jurdzinski and T.
%Henzinger. Quantitative simple stochastic partity games. In
%\emph{Proceeding of SODA'04}. SIAM.
%

%
%\bibitem{FV97} J. Filar and K. Vrieze. \emph{Competitive Markov
%Decision Processes}. Springer, 1997.
%
%\bibitem{Gri88} D. Grigoriev. Complexity of deciding Tarski
%algebra. \emph{Journal of Symbolic Computation}, 5(1-2):65-108,
%1988.
%
\bibitem{HJ94} H. Hansson and B. Jonsson. A Logic for reasoning
about time and reliability. \emph{Formal Aspects of Computing},
6(5): 512-535, 1994.
%
%%\bibitem{KS05} A. Kucera, O. Strazovsky. On the Controller Synthesis for
%%Finite-State Markov Decision Processes. In \emph{Proceeding of
%%FSTTCS'05}. LNCS 3821, 541-552, Springer, 2005.
%

%
%\bibitem{KVW01} O. Kupferman, M. Vardi and P. Wolper. Module checking. \emph{Information and
%Computation}. 164(2): 322-344, 2001.

\bibitem{Mar98} D. Martin. Then determinacy of blackwell games.
{\em Jouranl of  Symbolic Logic}, 63(4):1565-1581, 1998.



%
%%\bibitem{Lam80} L. Lamport. Sometimes is sometimes ``not never" -
%%on the temporal logic of program. In \emph{Proc. 7th ACM POPL},
%%174-185, 1980.
%

\bibitem{vNM47} J. van Neumann and O. Morgenstern. {\em Theory of
Games and Economic Behavior}. Princeton University Press,
Princeton, 1947.


\bibitem{Owe95} G. Owen. \emph{Game Theory}. Academic Press, 1995.
%
\bibitem{Put94} M. Puterman. \emph{Markov Decision Processes}.
Wiley, 1994.
%
%

\bibitem{Pit06} %Nir Piterman:

N. Piterman. From nondeterministic buchi and streett automata to
deterministic parity automata. In Proc. LICS 2006, pp. 255-264,
IEEE Computer Society, 2006.

\bibitem{Pnu77} A. Pnueli. The temporal logic of pragrams. In
Proc. FOCS 1977, pp. 46-57. IEEE Computer Society, 1977.

%\bibitem{Pnu85} A. Pnueli. Linear and branching structures in
%the semantics and logics of reactive systems. In \emph{Proc.
%ICALP'85}. LNCS 194, 15-32, Springer, 1985.

\bibitem{Pap94} C. Papadimitriou. \emph{Computational Complexity}.
Addison-Wesley, 1994.

\bibitem{QS82} J. Queille and J. Sifakis. Specification and
verification of concurrent systems in CESAR. In Proc. 5th {\em
International Symposium on Programming}, LNCS 137, pp. 337-351,
Springer, 1982.

\bibitem{RW89} P. Ramadge and W. Wonham. The control of discrete
event systems. {\em Proceedings of the IEEE}, 77(1): 81-89, 1989.

\bibitem{Sch04} P. -Y. Schobbens. Alternating-time logic with
imperfect recall. In Proc. LCMAS 2003, ENTCS 85, Elsevier, 2004.

\bibitem{SF06} S. Schewe and B. Finkbeiner. Satisfiability and Finite Model
Property for the Alternating-Time mu-Calculus. In Proc. CSL 2006,
LNCS 4207, pp. 591-605, Springer, 2006.

\bibitem{SSM06} M. Slanina, H. Sipma, and Z. Manna. Proving
ATL$^*$ Properties of Infinite-State Systems. In Proc. ICTAC 2006,
LNCS 4281, pp. 242-256, 2006.

\bibitem{Tar51} A. Traski. \emph{A Decision Method for Elementary
Algebra and Geometry}. Univ. of California Press, Berkeley, 1951.
%
%\bibitem{Tho03} W. Thomas. Infinite games and verification.
%In \emph{Proceeding of CAV'03}, LNCS 2725, 58-64, Springer, 2003.
\bibitem{Tho97} W. Thomas. Languages, automata, and logic. In G. Rozenberg and A.
Salomaa, editors, {\em Handbook of Formal Languages}, volume III,
pp. 389-455. Springer, 1997.

\bibitem{Wal04} I. Walukiewics. A landscape with games in the
background. In Proc. LICS 2004, pp. 356-366. IEEE Computer
Society, 2004.


\end{thebibliography}







\end{document}
